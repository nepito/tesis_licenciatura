{\chapter{Modelo Est\'andar}}
\section{Introducci\'on}
El modelo de las interacciones de las part\'iculas elementales hoy visto como 
est\'andar es el resultado de una larga construcci\'on y muchas contribuciones 
ingeniosas, entre las cuales las de  S.L. Glashow (1961), S. Weinberg (1967), y 
A. Salam (1967) son especialmente notables ~\cite{Giu200701}. En el marco del 
modelo est\'andar la materia del universo est\'a hecha de fermiones elementales 
interactuando a trav\'es de campos, de los cuales ellos son la fuente. Las 
part\'iculas asociadas con las interacciones de los campos son bosones. El 
modelo est\'andar describe las interacciones electromagn\'eticas, fuerte y 
d\'ebil de las part\'iculas elementales, y su construcci\'on fue guiada por un 
principio de simetr\'ia. La fuerza gravitacional no es incluida en el modelo
est\'andar.

En el presente trabajo se estudian las matrices de masa que mejor logren ajustar
la matriz de mezcla $V_{ckm}$ a los datos experimentales. La teor\'ia f\'isica
con la que se calcular\'an los observables te\'oricos que se contrastar\'an con 
los datos experimentales es el modelo est\'andar electrod\'ebil. Entre los datos
experimentales con los que se comparar\'a est\'an los producidos por el 
fen\'omeno de violaci\'on de $CP$, tema que se trata en la siguiente secci\'on y
para el cual se necesita la lagrangiana del modelo est\'andar. 

En esta secci\'on se presentan las partes que conforman a la densidad 
lagrangiana del modelo est\'andar, explicando brevemente su papel en \'este. Una
vez presentes las partes, la secci\'on 1 termina con la construcci\'on de la
densidad lagrangiana del modelo est\'andar y se comenta que los acoplos de los
campos de los quarks con los campos de norma y los acoplos de Yukawa tienen que
ser obtenidos del experimento, el modelo est\'andar no predice un valor para
ellos. En la primera
parte se hace una breve introducci\'on al modelo est\'anadar y la lagrangiana 
cin\'etica de los bosones de norma. En las dos siguentes partes se presentan el
resto de los t\'erminos de la lagrangiana, la lagrangiana din\'amica de los
fermiones y los t\'erminos de los que se obtienen los campos masivos. La
secci\'on acaba con la cuarta parte, en donde se generaliza lo visto en las tres
anteriores para poder obtener la lagrangiana del modelo est\'andar para tres
generaciones de quarks y leptones. 

\subsection{Lagrangiana cin\'etica de los bosones}
El modelo est\'andar de las interacciones fundamentales es la teor\'ia 
que describe las interacciones fuerte, d\'ebil y electromagn\'etica de las
part\'iculas elementales (quarks y leptones). El modelo est\'andar es una
teor\'ia de norma basada en el grupo de simetr\'ia $SU(3)_c\times\gme$ 
~\cite{Cot200701}. Es posible estudiar solamente la interacci\'on
 electrod\'ebil del modelo basada en el grupo de simetr\'ia
$SU(2)_L\times U(1)_Y$, esto se debe a que la simetr\'ia del color $SU(3)_C$
no se rompe, por lo que no se mezcla con el grupo \gme. Los grupos determinan
las interacciones y el n\'umero de bosones de norma
\footnote{Los bosones de norma son las part\'iculas encargadas de mediar las
interacciones.}, un bos\'on de norma por cada uno de los generadores del grupo,
el n\'umero y las propiedades de los bosones escalares y los fermiones es libre,
pero ellos deben transformarse de una forma definida por el grupo de simetr\'ia.
La parte electrod\'ebil tiene cuatro bosones de norma, tres de ellos tienen masa
y el otro no, que corresponden a los tres generadores del grupo $SU(2)_L$ y al 
generador del grupo $U(1)_Y$. La densidad lagrangiana del modelo est\'andar 
electrod\'ebil (ME) es la suma de las contribuciones de los fermiones 
($\mathcal{L}_f$) y los bosones de norma ($\mathcal{L}_b$)
\be\label{1.1}
\mathcal{L}=\mathcal{L}_f+\mathcal{L}_b.
\ee 
La densidad lagrangiana ~(\ref{1.1}) debe de cumplir con ciertas propiedades:
tiene que ser invariante local de transformaciones del grupo \gme, reproducir la
fenomenolog\'ia de la interacci\'on electrod\'ebil y, adem\'as, los fermiones y 
tres de los cuatro bosones de norma deben ser masivos. 

%Cualquier teor\'ia de part\'iculas elementales debe ser consistente con la 
%relatividad especial. La combinaci\'on de la mec\'anica cu\'antica, el 
%electromagnetismo y la relatividad especial da la ecuaci\'on de Dirac y la
%cuantizaci\'on de campos en teor\'ia cu\'antica de campos ~\cite{Cot200701}. 
El primer triunfo de la teor\'ia cu\'antica de campos es la electrodin\'amica
cu\'antica ~\cite{Cot200701}, la cual describe las interacciones del electr\'on
y el campo electromagn\'etico. El modelo est\'andar es una teor\'ia de
interacci\'on de campos. Para estudiar la fenomenolog\'ia se ve en la densidad
lagrangiana del modelo est\'andar que existen t\'erminos para cada una de las
interacciones. En el ME se unifican las interacciones electromagn\'eticas y
d\'ebil; la unificaci\'on significa que son dos diferentes manifestaciones de
una misma interacci\'on. 

En analog\'ia con la energ\'ia cin\'etica del fot\'on, los t\'erminos 
cin\'eticos para los bosones de norma son 
\be\label{bono}
\mathcal{L}_b=-\frac{1}{4}B_{\mu\nu}B^{\mu\nu}-\sum^3_{i=1}\frac{1}{4}
W^i_{\mu\nu}W^{i\mu\nu}
\ee
en donde $B_{\mu\nu}$ es el campo de fuerza para el campo de norma $U(1)_Y$ dado
por
\be\label{tf}
B_{\mu\nu}=\partial_{\mu}\hat B_{\nu}-\partial_{\nu}\hat B_{\mu}
\ee
y $W^i_{\mu\nu}$ es la generalizaci\'on de ~(\ref{tf}) para grupos no abelianos,
es decir
\be\label{tfw}
W^i_{\mu\nu}=\partial_{\mu}\hat W^i_{\nu}-\partial_{\nu}\hat W^i_{\mu}+g
\epsilon_{jki}\hat W^j_{\mu}\hat W^k_{\nu}.
\ee
y se expresan en t\'erminos de los tres bosones de norma del grupo $SU(2)_L$, 
$\hat {\bf W}_{\mu}$, y el bos\'on de norma de $U(1)_Y$, $B_{\mu}$ 
~\cite{Giu200701}.


%%%%%%%%%%%%%%%%%%%%%%%%%%%%%%%%%%%%%%%%%%%%%%%%%%%
\subsection{Lagrangiana din\'amica de fermiones}
La parte din\'amica de los fermiones se obtiene imponiendo la invariancia de 
norma a la ecuaci\'on de Dirac de un fermi\'on sin masa bajo transformaciones 
del grupo \gme ~\cite{Cot200701} la cual es
\be\label{ldf}
\mathcal{L}^{din}_f=\bar\psi_Ri\gamma^{\mu}D_{\mu}\psi_R\bar \psi_Li\gamma^{\mu}
D_{\mu}\psi_L.
\ee
donde para el grupo $\gme$ la derivada covariante $D_{\mu}$, la cual en las
teor\'ias de norma reemplaza a la derivada ordinaria $\partial_{\mu}$ en la
lagrangiana, tiene la forma
\be\label{dcme}
\hat D_{\mu}=\partial_{\mu}+ig_2{\vec\tau}\cdot\frac{\hat {\bf W}_{\mu}}{2}
+ig_1Y\frac{\hat B_{\mu}}{2},
\ee
para los dobletes de quiralidad izquierda L de $SU(2)$, y la forma
\be\label{dcme2}
\hat D_{\mu}=\partial_{\mu}+ig'Y\frac{\hat B_{\mu}}{2}
\ee
para los singletes de quiralidad derecha R.
En la ecuaci\'on ~(\ref{ldf}) se introdujo la notaci\'on de quiralidad de la 
ecuaci\'on de Dirac ~\cite{Cot200701}, donde $\psi$ es un espinor de Dirac de
cuatro componentes, las quiralidades definidas por
$$
\psi_R=\frac{1}{2}(1+\gamma^5)\psi\quad\psi_L=\frac{1}{2}(1-\gamma^5)\psi,
$$ 
$\bar\psi=\psi^{\dag}\gamma^0$, $\gamma^5=i\gamma^0\gamma^1\gamma^2\gamma^3$ y $\gamma^{\mu}$ con $\mu=0,1,2,3$
son las matrices de Dirac\footnote{Para una explicaci\'on detallada ver 
~\cite{Ait200401}, ~\cite{Cot200701} o ~\cite{Gri198701}.}. Sustituyendo
~(\ref{dcme}) y ~(\ref{dcme2}) en ~(\ref{ldf}) para la parte izquierda y
derecha, respectivamente, se obtienen los acoplos de los fermiones con los
bosones de norma, de aqu\'i se obtienen las interacciones electrod\'ebiles. Para
una familia de quarks, el t\'ermino de ~(\ref{ldf}) correspondiente a las partes
izquierdas de los espinores de los quarks tiene la forma 
\be
\mathcal{L}^{din,L}_f=\left(\bar u_L,\bar d_L\right)i\gamma^{\mu}
(\partial_{\mu}+ig_2{\vec\tau}\cdot\frac{\hat {\bf W}_{\mu}}{2}+ig_1Y
\frac{\hat B_{\mu}}{2})\left(\ba{c} u_L\\ d_L\ea\right),
\ee
donde el primer t\'emino de la derecha es la eciaci\'on de Dirac de part\'icula
libre y los t\'erminos restantes son los de interacci\'on con los campos de
norma, aqu\'i ${\vec \tau}$ son las matrices de esp\'in de Pauli siguientes
$$
\tau_1=\left(\begin{array}{lr} 0 & 1\\ 1 & 0 \end{array}\right) \qquad
\tau_2=\left(\begin{array}{lr} 0 & -i\\ i & 0 \end{array}\right) \qquad
\tau_3=\left(\begin{array}{lr} 1 & 0\\ 0 & -1 \end{array}\right).
$$
Los acoplos del espinor fermi\'onico con los campos de norma $\hat{\bf W}_{\mu}$
son
$$
g_2j^{\mu}_1W^1_{\mu}=\frac{g_2}{2}\left(\bar u_L,\bar d_L\right)
\gamma^{\mu}\tau_1\left(\ba{c}u_L\\ d_L\ea\right)W^1_{\mu},
$$
$$
g_2j^{\mu}_2W^2_{\mu}=\frac{g_2}{2}\left(\bar u_L,\bar d_L\right)
\gamma^{\mu}\tau_2\left(\ba{c} u_L\\ d_L\ea\right)W^2_{\mu}
$$
y
$$
g_2j^{\mu}_3W^3_{\mu}=\frac{g_2}{2}\left(\bar u_L,\bar d_L\right)
\gamma^{\mu}\tau_3\left(\ba{c} u_L\\ d_L\ea\right)W^3_{\mu},
$$
donde se observa que las dos primeras ecuaciones acoplan al bos\'on de norma
componentes distintos del doblete. Esta caracter\'istica juega un papel muy
importante en la violaci\'on de $CP$ y se explicar\'a en la siguiente secci\'on.

Utilizando las matrices de Pauli, ${\vec\tau\cdot}\hat{\bf W}_{\mu}$ toma la 
forma
$$
{\vec\tau\cdot}\hat{\bf W}_{\mu}
=\left(\begin{array}{lr} W^3_{\mu} & W^1_{\mu}-iW^2_{\mu}\\
W^1_{\mu}+iW^2_{\mu} & -W^3_{\mu} \end{array}\right),
$$
con lo cual las componentes de la corrientes del isoesp\'in d\'ebil pueden
expresarse de una manera compacta
\be\label{3cid}
{\bf J}^{\mu}=\frac{1}{2}\left(\bar u_L,\bar d_L\right)\gamma^{\mu}
\vec\tau\left(\ba{c} u_L\\ d_L\ea\right),
\ee
as\'i que la forma compacta de las corrientes de isoesp\'in d\'ebil acopladas a 
los campos de norma de $SU(2)_L$ toma la forma
\be\label{cnd}
g_2{\bf J}^{\mu}\hat{\bf W}_{\mu}=\frac{g_2}{2}\left(\bar u_L,d_L\right)
\gamma^{\mu}\vec\tau\left(\ba{c} u_L\\ d_L\ea\right)\hat{\bf W}_{\mu}.
\ee
Para el acoplo del espinor con el campo de norma $\hat B_{\mu}$ se utiliza la
relaci\'on de Gell-Mann-Nishijima para expresar la hipercarga en funci\'on de la
carga el\'ectrica y la tercera componente del isoesp\'in
\be\label{rgmn}
Q=\tau_3+\frac{Y}{2}
\ee
finalmente se tiene
\be\label{cyl}
\frac{g_1}{2}j^{\mu}_{Y,L}\hat B_{\mu}=g_1\left(\bar u_L,\bar d_L\right)
\gamma^{\mu}(Q-\tau_3)\left(\ba{c} u_L\\ d_L\ea\right)\hat B_{\mu}.
\ee
De igual manera, para la parte de quiralidad derecha de ~(\ref{ldf}) utilizando
~(\ref{dcme2}) se tiene
\be\label{cyr}
\frac{g_1}{2}j^{\mu}_{Y,R}\hat B_{\mu}=g_1\left(\bar u_R,\bar d_R\right)
\gamma^{\mu}(Q-\tau_3)\left(\ba{c} u_R\\ d_R\ea\right)\hat B_{\mu}.
\ee
Las ecuaciones ~(\ref{cyl}), ~(\ref{dcme2}) y la tercera componente de la 
corriente d\'ebil acoplan los bosones de norma a una misma componente del
doblete. Dividiendo la densidad lagrangiana din\'amica ~(\ref{ldf}) en dos, una
parte de interacci\'on y una parte cin\'etica, es posible expresarla utilizando 
 ~(\ref{cnd}), ~(\ref{cyl}) y ~(\ref{cyr}) como
\be\label{17}
\mathcal{L}^{din}_f=L^{cin}_f+g_2{\bf J}^{\mu}\hat{\bf W}_{\mu}
+\frac{g_1}{2}(j^{\mu}_{Y,R}+j^{\mu}_{Y,L})\hat B_{\mu}.
\ee
Todas las interacciones existentes en el ME de los bosones de norma con los 
fermiones est\'an presentes en ~(\ref{17}).


%%%%%%%%%%%%%%%%%%%%%%%%%%%%%%%%%%%%%%%
\subsection{Generaci\'on de masa}
Para proporcionar masa a los bosones y fermiones se utliza el mecanismo de
Higgs, con el cual se induce un rompimiento espont\'aneo de simetr\'ia.  
Hasta el momento los cuatro bosones de norma ($\hat{\bf  W}_{\mu}$ y $\hat 
B_{\mu}$) y los fermiones se mantienen sin masa. Para resolver esto se agrega
la energ\'ia cin\'etica y potencial (la densidad lagrangiana) del campo de 
Higgs, $\Phi$
\be\label{lhiggs}
\mathcal{L}_{\Phi}=(D_{\mu}\Phi)^{\dag}(D^{\mu}\Phi)-V(\Phi^{\dag}\Phi).
\ee
Despu\'es del rompimiento de simetr\'ia el campo de Higgs
proporciona masa a los bosones de norma. 

El campo de Higgs es un doblete de $SU(2)$
$$
\Phi=\left(\begin{array}{c}\phi^+\\ \phi^0\end{array}\right),
$$
con un valor de expectaci\'on del vac\'io dado por
$$
\langle\Phi\rangle_0=\left(\ba{c} 0\\ v
\ea\right)
$$
y con estados excitados de la forma
\be\label{vevhe}
\left(\ba{c} 0\\ v+\frac{h(x)}{\sqrt{2}}  \ea\right)
\ee
donde $h$ es el campo f\'isico del Higgs. La lagrangiana de Higgs debe de ser
invariante de norma localmente. Si el valor de expectaci\'on del vac\'io del
campo de Higgs no es invariante, se dice que hay un rompimiento espont\'aneo de 
simetr\'ia. 

Para que los fermiones obtengan masa se introduce un t\'ermino de interacci\'on
entre el campo de Higgs y los fermiones, que debe ser invariante de norma, 
respetar la simetr\'ia del grupo $\gme$ y del que se obtengan masas de Dirac. La
lagrangiana de Yukawa
\be\label{ly}
\mathcal{L}_{Yuk}=-G_{\psi}\left[(\psi^{\dag}_L\Phi)\psi_R+
\psi^{\dag}_R(\Phi^{\dag}\psi_L)\right],
\ee
cumple con las condiciones requeridas. Despu\'es del rompimiento de simetr\'ia, 
el campo de Higgs $\Phi$ da masa a los fermiones
$$
\mathcal{L}_{Yuk}=-vG_{\psi}\left[\psi^{\dag}_L\psi_R+
\psi^{\dag}_R\psi_L\right],
$$
y se reconoce la forma de las masas de Dirac
\be\label{jojojo}
m_{\psi}=vG_{\psi}.
\ee
En el modelo est\'andar el mecanismo de Higgs es el responsable de que los 
campos de norma y los fermiones adquieran masa. Los detalles aun son 
especulaci\'on pues la part\'icula de Higgs nunca se ha visto y el potencial de 
Higgs es completamente desconocido, ~(\ref{phdres}) es solo una propuesta 
~\cite{Gri198701}.
Con esto es posible hacer que la densidad lagrangiana del ME est\'e completa; 
se tienen todas las interacciones del ME, el fot\'on se mantiene sin masa, los
fermiones y tres bosones de norma adquieren masa.

%%%%%%%%%%%%%%%%%%%%%%%%%%%%%%%%%%%%%%%%%%%%%
\subsection{Modelo Est\'andar con tres familias}
Para obtener la lagrangiana del ME se trabajar\'a con seis dobletes de 
quiralidad izquierda L, tres corrspondientes a los leptones
\be
L_1\equiv\left(\ba{c} \nu_{eL}\\ e_L  \ea\right), \qquad
L_2\equiv\left(\ba{c} \nu_{\mu L}\\ \mu_L  \ea\right), \qquad
L_3\equiv\left(\ba{c} \nu_{\tau L}\\ \tau_L  \ea\right),
\ee
y tres correspondientes a los quarks
\be
Q_1\equiv\left(\ba{c} u_L\\ d_L  \ea\right), \qquad
Q_2\equiv\left(\ba{c} c_L\\ s_L  \ea\right), \qquad
Q_3\equiv\left(\ba{c} t_L\\ b_L  \ea\right), \qquad
\ee
y con los correspondientes singletes de quiralidad derecha R
$$
l_e\equiv e_R,\qquad
l_{\mu}\equiv \mu_R,\qquad
l_{\tau}\equiv \tau_R,\qquad
$$
$$
l_{\nu_e}\equiv \nu_{eR},\qquad
l_{\nu_\mu}\equiv \nu_{\mu R},\qquad
l_{\nu_\tau}\equiv \nu_{\tau R},\qquad
$$
$$
u_{1}\equiv u_R,\qquad
u_{2}\equiv c_R,\qquad
u_{3}\equiv t_R,\qquad
$$
$$
d_{1}\equiv d_R,\qquad
d_{2}\equiv s_R,\qquad
d_{3}\equiv b_R.\qquad
$$
La generalizaci\'on de la lagrangiana de Yukawa consta de dos partes, una 
lagrangiana para los quarks
\be\label{lygq}
\mathcal{L}^q_{Yuk}=-\sum_{ij}\left[Y^D_{ij}(Q_{i}^{\dag}\Phi)d_{j}+
Y^{D*}_{ij}d^{\dag}_{j}(\Phi^{\dag}Q_{i})\right]
-\sum_{ij}\left[Y^U_{ij}(Q_{i}^{\dag}\tilde\Phi)u_{j}+
Y^{U*}_{ij}u^{\dag}_{j}(\tilde\Phi^{\dag}Q_{i})\right],
\ee
donde $i,j=1,2,3$ por lo que $Y^q_{ij}$ son los nueve elementos de una matriz
compleja de $3\time 3$; y otra para los leptones
\be\label{lygl}
\mathcal{L}^l_{Yuk}=-\sum_{i\beta}\left[Y^e_{i\beta}(L_{i}^{\dag}\Phi)l_{\beta}+
Y^{e*}_{i\beta}l^{\dag}_{\beta}(\Phi^{\dag}Q_{i})\right]
-\sum_{i\alpha}\left[Y^{\nu}_{i\alpha}(L_{i}^{\dag}\tilde\Phi)l_{\alpha}+
Y^{\nu*}_{i\alpha}l^{\dag}_{\alpha}(\tilde\Phi^{\dag}L_{i})\right],
\ee
donde $\beta=e,\mu,\tau$ y $\alpha=\nu_e,\nu_{\mu},\nu_{\tau}$ y los acoplos de 
Yukawa son matrices complejas y arbitrarias; entonces, la lagrangiana de Yukawa
es
\be\label{lyg}
\mathcal{L}_{Yuk}=\mathcal{L}^q_{Yuk}+\mathcal{L}^l_{Yuk}.
\ee
La lagrangiana din\'amica de los fermiones ~(\ref{ldf}) para el ME es
%\be\label{ldg}
$$
\mathcal{L}_{din}=i\sum_{1,2,3}\left(\bar L_i\gamma^{\mu}D_{\mu}L_i+
\bar Q_i\gamma^{\mu}D_{\mu}Q_i+\bar d_i\gamma^{\mu}D_{\mu}d_i +
\bar u_i\gamma^{\mu}D_{\mu}u_i\right)
$$
\be\label{ldg}
+i\sum_{\alpha}\bar l_{\alpha}\gamma^{\mu}D_{\mu}l_{\alpha}+i\sum_{\beta}\bar 
l_{\beta}\gamma^{\mu}D_{\mu}l_{\beta}.
\ee



La densidad lagrangiana de ME para tres generaciones es la suma de 
~(\ref{ldg}), ~(\ref{bono}), ~(\ref{lhiggs}) y ~(\ref{lyg})
\be\label{lme3f}
\mathcal{L}_{ME}=\mathcal{L}_{din}+\mathcal{L}_b+\mathcal{L}_{\Phi}
+\mathcal{L}_{Yuk}.
\ee
En el ME no hay forma de predecir al valor de los acoplos de norma $g_1$ y 
$g_2$, tampoco de las masas de los fermiones, la forma de los acoplos de Yukawa
no es predicha por el modelo est\'andar, ni del bos\'on de Higgs ni
del acoplo cu\'artico. Todos estos datos tiene que obtenerse del experimento
y con ellos ajustar el modelo para que haga una reproducci\'on de la realidad lo
m\'as precisa posible. Hasta ahora se trabajo con la lagrangiana inavariante de norma, las matrices de masa de los fermiones no representan las masas f\'isicas,
para que representen las masas f\'isicas se deb pasar a la bases donde las
matrices son diagonales y los eigenvalores son las masas de los fermiones. Uno
de los fen\'omenos que debe reproducir la densidad lagrangiana ~(\ref{lme3f}) es
la violaci\'on de $CP$. En la siguiente secci\'on se presentan las reglas de
transformaci\'on de los campos y se prueba la simetr\'ia de las partes de 
~(\ref{lme3f}) ante dichas transformaciones.
%%%%%%%%%%%%%%%%%%%%%%%%%%%%%%%%%%%%%%%%%%%%%%%%%%%%%%%%%%%%%%%%%%%%%%%%%%%%%5
%%%%%%%%%%%%%%%%%%%%%%%%%%%%%%%%%%%%%%%%%%%%%%%%%%%%%%%%%%%%%%%%%%%%%%%%%%%
%%%%%%%%%%%%%%%%%%%%%%%%%%%%%%%%%%%%%%%%%%%%%%%%%%%%%%%%%%%%%%%%%%%%%%%%%%

\section{Violaci\'on de $CP$}
Las leyes de conservaci\'on en f\'isica son debidas a la invariancia de los
sistemas bajo transformaciones de simetr\'ia. La invariancia ante la paridad 
significa que la derecha y la izquierda no pueden ser definidas en una esencia 
absoluta. La ley de invariancia de la conjugaci\'on de carga significa que los 
experimentos en un mundo de antimateria dar\'an resulatados equivalentes que los
hechos en este mundo.

En f\'isica cl\'asica, la paridad no es afectada por la coordenada temporal, es 
decir, conmuta con los desplazamientos temporales $t\rightarrow t+\Delta 
t$. Siguiendo el principio de correspondencia, para pasar de mec\'anica 
cl\'asica a mec\'anica cu\'antica, se requiere la representaci\'on mec\'anico 
cu\'antica del operador $\cal P$ y de la traslaci\'on temporal, a trav\'es del 
operador  $e^{-i\mathcal{H}\Delta t}$, para reproducir esta caracter\'istica. El
operador $\cal P$ debe conmutar con el operador $\mathcal{H}$, ya que en la 
teor\'ia cu\'antica ellos conmutan, lo que significa que la paridad es una buena
simetr\'ia en la naturaleza. La paridad no es un simetr\'ia en las interacciones
d\'ebiles; los experimentos de los decaimientos nucleares $\beta$, $\pi^{\pm}$ y
$\mu^{\pm}$ demostraron la violaci\'on  de $P$ e invariancia de $C$ bajo las 
interacciones d\'ebiles, por lo que no se debe satisfacer el requerimiento de 
ser una buena simetr\'ia. Una conclusi\'on equivalente se obtiene para el 
operador $\mathcal{C}$. Entonces, los operadores $\mathcal{C}$ y $\mathcal{P}$ 
no son los mismos a los definidos en mec\'anica cu\'antica, por lo que para 
definir las reglas de transformaci\'on de $CP$ se utiliza la parte de la 
lagrangiana que debe respetar la simetr\'ia. Suponiendo que el electromagnetismo
es invariante ante transformaciones $C$ y $P$, como sugieren los experimentos, 
es posible definir las reglas de transformaci\'on a partir de la lagrangiana de 
la electrodin\'amica y extenderlas al resto de la lagrangiana electrod\'ebil. 


Para que la lagrangiana del ME sea consistente con los experimentos debe tener 
al menos un t\'ermino que viole la simetr\'ia discreta $CP$. Tomando lo anterior
en cuenta la lagrangiana completa puede ser escrita como la suma de dos 
t\'erminos
$$
\mathcal{L}=\mathcal{L}_{CP}+\mathcal{L}_R
$$
donde $\mathcal{L}_{CP}$ es la parte en la que se conserva $CP$ y ${\cal L}_R$
la parte que viola $CP$. La manera de incluir la violaci\'on de $CP$ es en la
 matriz de mezcla, la cual no puede obtenerse de la teor\'ia, sino de los datos
experimentales. La matriz de mezcla juega un papel importante en el presente 
trabajo, es la fuente de la mayor\'ia de los datos que se usan como criterio de
la consistencia de los modelos te\'oricos estudiados. Los modelos te\'oricos 
deben ajustar los m\'odulos de los elementos de la matriz de mezcla, los 
\'angulos internos de los tri\'angulos unitarios y el invariante de Jarlskog. 

En esta secci\'on se habla de la violaci\'on de $CP$, no se deducen las reglas
de transformaci\'on, pero se transforma a la lagrangiana del ME utilizando estas
reglas. 

%%%%%%%%%%%%%%%%%%%%%%%%%%%%%%%%%%%%%%%%%%%%%%%%%%%%%%%%%%%% 
\subsection{Transformaciones de $CP$ en la lagrangiana del ME}
Las reglas de transformaci\'on de $CP$, de los campos en la lagrangiana del ME, 
son definidas por la parte de la lagrangiana que conserva $CP$, y est\'an dadas 
en el cuadro 1.
\begin{table}[h!]\label{t1}
\caption{Transformaciones discretas de campos y lo campos bilineales de Dirac.}
$$\ba{c|ccccc}
Campo & P & T & C & CP & CPT\\ \hline
B^{\mu}&  &   &   & -B_{\mu}& \\
W^{1\mu}&W^1_{\mu}& & -W^{1\mu} & -W_{1\mu} & \\
W^{2\mu}&W^2_{\mu}& & W^{2\mu} & W_{2\mu} & \\
W^{3\mu}&W^3_{\mu}& & -W^{3\mu} & -W_{3\mu} & \\
W^{+\mu}& &   &   & -e^{i\xi_w}W^-_{\mu} & \\
W^{-\mu}& &   &   & -e^{-i\xi_w}W^+_{\mu} & \\
A^{\mu}&  &   &   & -A_{\mu}& \\
Z^{\mu} & &   &   & -Z_{\mu} & \\
h       & &   &   & h        & \\
\bar\psi\xi & \bar\psi\xi & \bar\psi\xi & \bar\xi\psi & \bar\xi\psi & 
\bar\xi\psi\\

\bar\psi\gamma_5\xi & -\bar\psi\gamma_5\xi & \bar\psi\gamma_5\xi & \bar\xi\gamma
_5\psi & -\bar\xi\gamma_5\psi & -\bar\xi\gamma_5\psi\\

\bar\psi\gamma^{\mu}\xi & \bar\psi\gamma_{\mu}\xi & \bar\psi\gamma_{\mu}\xi & 
-\bar\xi\gamma^{\mu}\psi & -\bar\xi\gamma_{\mu}\psi & -\bar\xi\gamma^{\mu}\psi\\

\bar\psi\gamma^{\mu}\gamma_5\xi & -\bar\psi\gamma_{\mu}\gamma_5\xi & \bar\psi
\gamma_{\mu}\gamma_5\xi & \bar\xi\gamma^{\mu}\gamma_5\psi & -\bar\xi\gamma^{\mu}
\gamma_5\psi & -\bar\xi\gamma^{\mu}\gamma_5\psi\\ %\hline
\ea $$\end{table}

En el presente trabajo solo se considera el sector hadr\'onico (los quarks); el 
sector lept\'onico puede ser tratado en manera similar, si los neutrinos son 
part\'iculas de Dirac. Para mostrar qu\'e  partes de la lagrangiana respetan
$CP$ (son invariantes ante la acci\'on conjunta de los operadores ${\cal C}$ y
${\cal P}$) y qu\'e condiciones se deben cumplir para que exista violaci\'on de 
$CP$, se trabaja con la lagrangiana despu\'es del rompimiento espont\'aneo de 
simetr\'ia, con campos de norma y quarks masivos, y al diagonalizar la matriz de
masa, con los campos de los quarks f\'isicos, ya que hay partes de la 
lagrangiana que no respetan la simetr\'ia cuando no se trabaja con los campos 
f\'isicos y s\'i es respetada por los campos f\'isicos ~\cite{Ber199701}. 

Primero se trabaja con los t\'erminos que no contienen a los campos de los 
fermiones. Tomando en cuenta que una lagrangiana pura de norma es invariante 
ante las transformaciones de $CP$ ~\cite{Gri199501}, se muestra que el t\'ermino
cin\'etico de los bosones de norma es invariante de $CP$. Aplicando las reglas 
de transformaci\'on del cuadro 1 a las ecuaciones ~(\ref{tf}) y ~(\ref{tfw})
$$B_{\mu\nu}=\partial_{\mu}\hat B_{\nu}-\partial_{\nu}\hat B_{\mu}$$
$$W^i_{\mu\nu}=\partial_{\mu}W^i_{\nu}-\partial_{\nu}W^i_{\mu}+g\epsilon_{jki}
W^j_{\mu}W^k_{\nu}$$
y observando que la derivada parcial se transforma bajo la acci\'on de 
${\cal CP}$ de la forma $\partial^{\mu}\rightarrow\partial_{\mu}$, se obtienen 
las transformaciones de los campos de fuerzas
\be\label{tcptf}
B^{\mu\nu}\fcp-B_{\mu\nu}
\ee
y
\be\label{tcptfw}
\left(W^1_{\mu\nu},W^2_{\mu\nu},W^3_{\mu\nu}\right)\fcp
-\left(W^{1\mu\nu},-W^{2\mu\nu},W^{3\mu\nu}\right).
\ee
Aplicando ~(\ref{tcptf}) y ~(\ref{tcptfw}) a la ecuaci\'on ~(\ref{bono}) se 
demuestra que la lagrangiana cin\'etica de los bosones es invariante ante la 
acci\'on de ${\cal CP}$

\be\label{invCPlb}
{\cal L}_b\fcp{\cal L}_b.
\ee

De manera equivalente para el sector del Higgs se observa que, despu\'es del 
rompimiento de simetr\'ia, el potencial de Higgs ~(\ref{phdres}) es invariante 
ante $CP$. Aplicando las reglas de transformaci\'on del cuadro 1 al resto de la
lagrangiana del Higgs ~(\ref{lh}) se muestra la invariancia de la lagrangiana
completa del Higgs
\be\label{invCPlh}
{\cal L}_{\Phi}\fcp{\cal L}_{\Phi}.
\ee
Los t\'erminos que no incluyen los campos de los fermiones son invariantes ante 
transformaciones de $CP$, el resto de los t\'erminos de la lagrangiana del ME 
~(\ref{lme3f}) involucran los campos f\'isicos de los fermiones (en particular
en este trabajo se estudiar\'a los campos de los quarks). Por lo anterior, la 
violaci\'on de $CP$ solo puede surgir de aquellos t\'erminos en los que est\'an 
presentes los campos f\'isicos de los quarks. 

Utilizando ~(\ref{vevhe}) y ~(\ref{jojojo}) en ~(\ref{lygq}) se obtiene la
lagrangiana de masa de los quarks, que son los t\'erminos de la lagrangiana de  
Yukawa despu\'es del rompimiento espont\'aneo de simetr\'ia que son 
proporcionales a $v$, y tiene la forma 
\be\label{lmq}
\mathcal{L}^q_{masa}=-\sum_{ij}\left[m^U_{ij}\bar u'_iu_j+m^{U*}_{ij}\bar u_j
u'_i\right]-\sum_{ij}\left[m^D_{ij}\bar d'_id_j+m^{D*}_{ij}\bar d_jd'_i\right],
\ee
en donde las $u_i$ son los singletes derechos de los quarks y los $u'_i$ son los
componentes de los dobletes izquierdos; las matrices de masa est\'an definidas 
por 
$$m^D_{ij}=vY^D_{ij}.$$
Aun no se est\'a trabajando con los campos f\'isicos de los quarks. Cuando se 
introdujo la lagrangiana de Yukawa se consider\'o que deb\'ia cumplir con 
ciertas condiciones de simetr\'ia (ser invariante de norma y respetar la 
simetr\'ia del grupo {\gme }), los acoplos de Yukawa, $Y^{U,D}$, m\'as generales
que cumplen con esas condiciones de simetr\'ia son matrices complejas. Para que 
los campos de los quarks $u_i$ representen campos f\'isicos, las matrices de 
masa deben ser diagonales y los valores de la diagonal ser las masas de los 
quarks, que se obtiene de los datos experimentales, recordando que en el ME no 
hay forma de predecir las masas de los fermiones. Como cualquier matriz 
cuadrada, sin importar si es hermitiana o no, puede ser diagonalizada mediante 
dos matrices unitarias, la matriz diagonal cuyos valores son las masas de los 
quarks es de la siguiente manera
\be\label{mdu}
U^{u\dag}_Lm^UU^{u}_R\equiv \mbox{diag}(m_u,m_c,m_t)
\ee
\be\label{mdd}
U^{d\dag}_Lm^DU^{d}_R\equiv \mbox{diag}(m_d,m_s,m_b).
\ee
Tal transformaci\'on es equivalente a cambiar los campos de los quarks de la
base de los eigenestados del sabor a la de los eigenestados de las masas, 
introduciendo acoplos no diagonales en la lagrangiana de las corrientes 
cargadas, en la cual solo las partes izquierdas est\'an involucradas.

Sustituyendo ~(\ref{mdu}) y ~(\ref{mdd}) en ~(\ref{lmq}) se obtiene 
\be\label{tlmq}
\bar u'_im^Uu_j=\bar u'mu=\bar u'U_LU^{\dag}_LmU_RU^{\dag}_Ru=
\overline{u^{fis}}'\mbox{diag}(m_u,m_c,m_t)u^{fis},
\ee
de donde los $u^{fis}_i$, los campos f\'isicos de los quarks tipo $u$ con masa
definida, son expresados de la siguiente manera

$$
u^{fis}_L=U^{q\dag}_L\left(\ba{c} u_L\\ c_L\\ t_L\ea\right).
$$
La lagrangiana de masa de los quarks f\'isicos es
$$
\mathcal{L}^{qf}_{masa}=-[m_u(\bar u_L^{fis}u_R^{fis}+
\bar u_R^{fis}u_L^{fis})]
-[m_d(\bar d_L^{fis}d_R^{fis}+\bar d_R^{fis}d_L^{fis})]
$$
\be\label{lmqf}
-[m_s(\bar s_L^{fis}s_R^{fis}+\bar s_R^{fis}s_L^{fis})]
-[m_c(\bar c_L^{fis}c_R^{fis}+\bar c_R^{fis}c_L^{fis})]
\ee
$$
-[m_b(\bar b_L^{fis}b_R^{fis}+\bar b_R^{fis}b_L^{fis})]
-[m_t(\bar t_L^{fis}t_R^{fis}+\bar t_R^{fis}t_L^{fis})],
$$ 
como de aqu\'i en adelante se trabajar\'a  con los campos de los quarks masivos 
se omitir\'a el super\'indice $fis$ a menos que se considere necesario para no
crear confusi\'on en la notaci\'on. Para mostrar que la lagrangiana de masa de
los quarks f\'isicos es invariante ante $CP$ se utilizan las reglas de 
transformaci\'on, mostradas en el cuadro 1, en el primer t\'ermino de 
~(\ref{lmqf}), obteniendo
$$
-[m_d(\bar d_Ld_R+\bar d_Rd_L)]\fcp-[m_d(\bar d_Rd_L+\bar d_Ld_R)]
$$
el resto de la lagrangiana ~(\ref{lmqf}) se transforman de manera 
equivalente. Los otros t\'erminos de la lagrangiana de Yukawa se trabajan igual,
utilizando la regla de transformaci\'on del campo de Higgs, por lo que se tiene
\be\label{incplY}
\mathcal{L}_{Yuk}\fcp\mathcal{L}_{Yuk}.
\ee

Para la parte din\'amica de la lagrangiana se hace un procedimiento equivalente.
Primero se trabaja con los t\'erminos de las corrientes neutras
$$
E\equiv i\bar u_LU^{u\dag}_L\gamma^{\mu}D_{\mu}U^{u}_Lu_L=i\bar u_L\gamma^{\mu}
\left[\partial_{\mu}+ig_2\tau_3\frac{\hat W^3_{\mu}}{2}
+ig_1Y\frac{B_{\mu}}{2}\right] u_L
$$
donde $u_L$, como ya se dijo, son los campos f\'isicos de los quarks $u$ de 
quiralidad izquierda. En la \'ultima igualdad se utiliz\'o la propiedad de las 
matrices unitarias, $U^{u\dag}_LU^u_L=1$. Dividiendo en tres partes a $E$,
expresando expl\'icitamante la matriz absorbida por la notaci\'on de la
quiralidad y aplicando las transformaciones de $CP$ se tiene
$$
E_1\equiv i\bar u\gamma^{\mu}\partial_{\mu}(1-\gamma^5)u\mathop{\rightarrow}
\limits^{CP}
(-i)\left[(-\bar u\gamma_{\mu}\partial^{\mu}u)-(-\bar u\gamma_{\mu}
\partial^{\mu}\gamma^5u)\right]
$$
$$
E_2\equiv \bar u\gamma^{\mu}g_2\frac{\hat W^3_{\mu}}{2}(1-\gamma^5)u
\fcp -\bar u\gamma_{\mu}g_2\left(\frac{-\hat W^{\mu}_3}{2}\right)(1-\gamma^5)u
$$
en donde hace falta un t\'ermino igual para los quarks tipo $d$ pero con signo 
contrario para $E_2$, causado por la tercera matriz de Pauli, $\tau_3$, y
$$
E_3\equiv \bar u\gamma^{\mu}g_1\left(\frac{2}{3}\right)\frac{B_{\mu}}{2}
(1-\gamma^5)u\fcp
-\bar u\gamma_{\mu}g_1\left(\frac{-B^\mu}{3}\right)(1-\gamma^5)u,
$$
con lo que se muestra la invariancia de los t\'erminos de las corrientes neutras
ante transformaciones de $CP$.

El \'ultimo t\'ermino del que puede surgir la violaci\'on de $CP$ es la parte de
las corrientes cargadas de la lagrangiana din\'amica ~(\ref{ldg}). Las 
corrientes cargadas son las \'unicas que restan por demostrar su invariancia. 
Como las corrientes cargadas solo involucran quarks parte izquierda, las 
matrices de los campos de los quarks f\'isicos parte derecha, $U^d_R$ y $U^d_R$,
no aparecen en las corrientes cargadas. El t\'ermino que falta transformar es de
las corrientes cargadas, que corresponde a las dos primeras componentes del
segundo t\'ermino, el de los quarks, de la ecuaci\'on ~(\ref{ldg})

$$
X_c\equiv[\hat W^1_{\mu}-i\hat W^2_{\mu}]\bar u_L\gamma^{\mu}U^{u\dag}_LU^d_Ld_L
+[\hat W^1_{\mu}+i\hat W^2_{\mu}]\bar d_L\gamma^{\mu}U^{d\dag}_LU^u_Lu_L,
$$
en donde la multiplicaci\'on de las matrices $U^{u\dag}_LU^d_L$ no 
necesariamente es la identidad. Aplicando las transformaciones de $CP$ a $X_c$ 
se obtiene
$$
X_c=[\hat W^1_{\mu}-i\hat W^2_{\mu}]\bar u_{\alpha}\gamma^{\mu}V_{\alpha j}
(1-\gamma_5)d_j+[\hat W^1_{\mu}+i\hat W^2_{\mu}]\bar d_j\gamma^{\mu}
V^*_{\alpha j}(1-\gamma_5)u_{\alpha}
$$
$$
\fcp[-\hat W^{1\mu}+i\hat W^{2\mu}](-\bar d_j\gamma_{\mu}V_{\alpha j}
(1-\gamma_5)u_{\alpha})+[-\hat W^{1\mu}-i\hat W^{2\mu}](-\bar u_{\alpha}
\gamma_{\mu}V^*_{\alpha j}(1-\gamma_5)d_j).
$$
As\'i, para que se conserve $CP$ la matriz $V_{ckm}=U^{u\dag}_LU^{d}_L$ debe de 
ser real. La \'unica manera para que el ME sea consistente con los datos 
experimentales, relacionados a la violaci\'on de la simetr\'ia discreta $CP$, es
que la matriz $V_{ckm}$ sea compleja, ya que solo el t\'ermino de la lagrangiana
del ME que puede no respetar la simetr\'ia $CP$ es el de las corrientes 
cargadas. No hay manera que la violaci\'on de $CP$ venga de alg\'un otro 
t\'ermino, como ya se vio en ~(\ref{invCPlb}), ~(\ref{invCPlh}) y 
~(\ref{incplY}).

En el siguiente cap\'itulo se presentan las propiedades de la matriz $V_{ckm}$ y
la utilizaci\'on de los tri\'angulos unitarios y el invariante de Jarlskog para
el estudio del fen\'omeno de la violaci\'on de $CP$.




