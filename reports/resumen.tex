\chapter*{Resumen}

En el presente trabajo se investiga la capacidad predictiva de cinco modelos de
texturas para las matrices de masa. La teor\'ia f\'isica con la que se calculan
los observables que se contrastan con los datos experimentales es el modelo
est\'andar electrod\'ebil. Los datos experimentales con los que se compara son
los modulos de la matriz de mezcla y  tres relacionados al fen\'omeno de
violaci\'on de $CP$.

Se usan algoritmos gen\'eticos para ajustarse a los datos experimentales de la 
mezcla de los quarks y la violaci\'on de $CP$. Mediante dicho ajuste se 
encuentran la matriz de mezcla de los quarks, los \'angulos internos del 
tri\'angulo unitario y el invariante de Jarlskog. El m\'etodo se implementa con
tres distintos conjuntos de intervalos de masa encontrados en la literatura 
calculados a energ\'ia de la masa del bos\'on $Z$ y cinco pares diferentes de
texturas para las matrices de masa. Se comparan los m\'odulos 
de los elementos de la matriz de mezcla y sus invariantes calculados contra
los reportados por Particle Data Group en el a\~no 2010, encontrando que 
\'unicamente una textura en las matrices de masa es compatible con toda la 
fenomenolog\'ia considerada. 
%Se pone a prueba la
%compatibilidad de las texturas y las masas con la suposic\'on del maximal en la
%violaci\'on de $CP$.
