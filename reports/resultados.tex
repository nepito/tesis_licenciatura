\chapter{Resultados}
\index{Resultados}

Se utiliz\'o un conjunto de masas en los intervalos reportados por Koide 
~\cite{Fus199701}, Xing ~\cite{xin200801} y Emmanuel-Costa ~\cite{Emm200901} 
que pudieran ajustarse a los datos experimentales de la mezcla de los quarks y
la violaci\'on de $CP$. Primero se hizo el ajuste con matrices de masa que 
presentaran en ambos sectores la textura de Fritzsch con dos ceros dada en la
ecuaci\'on ~(\ref{matfrit}), \'este es el modelo $F4$ presentado en la tabla 
4.1.
%~\ref{ttt1}. 
 
{\tiny
\begin{table}[h!]\label{ttt1}
\begin{scriptsize}
\caption{Matrices de {masa} hermitianas de los quarks puestas a prueba}
$$\ba{c|ccccc}\hline\hline
 & F4 & F2F3 & F3F2 & F2R1 & R1F2\\ \hline \\
M_u&\left(\ba{ccc} 0&A&0\\ A*&D&B\\ 0&B&C \ea\right) & 
\left(\ba{ccc} 0&A&0\\ A*&D&B\\ 0&B*&C \ea\right) & 
\left(\ba{ccc} 0&A&O\\ A*&0&B\\ 0&B*&C \ea\right) & 
\left(\ba{ccc} 0&A&0\\ A*&D&B\\ 0&B*&C \ea\right) & 
\left(\ba{ccc} 0&A&0\\ A*&D&0\\ 0&0&C \ea\right)\\  \\ \hline\\
M_d&\left(\ba{ccc} 0&A&0\\ A*&D&B\\ 0&B*&C \ea\right) & 
\left(\ba{ccc}     0&A&0\\ A*&0&B\\ 0&B*&C \ea\right) & 
\left(\ba{ccc}     0&A&0\\ A*&D&B\\ 0&B*&C \ea\right) & 
\left(\ba{ccc}     0&A&0\\ A*&D&0\\ 0&0&C \ea\right) & 
\left(\ba{ccc}     0&A&0\\ A*&D&B\\ 0&B*&C \ea\right)\\  \\ \hline
\ea $$
\end{scriptsize}
\end{table}
}


Se utiliz\'o un AG por 2160 generaciones para cada uno de los intervalos de la
dadas en la tabla ~\ref{t3masas}. Los resultados obtenidos para los m\'odulos de
la matriz de mezcla con el valor mayor de la funci\'on deseabilidad, $F_d$,
despu\'es de todas las generaciones son presentados en la tabla 4.2. En los
primeros 10 renglones de la tabla 4.2 se presentan los m\'odulos de la matriz de
mezcla te\'oricos obtenidos para cada uno de los tres conjuntos de intervalos de
masas y el cociente de los dos primeros m\'odulos del tercer rengl\'on en la
matriz $V_{ckm}$. En la \'ultima columna se presentan los valores actuales para
los m\'odulos de la matriz de mezclas ~(\ref{pdgvckm}) y reportados en 
~\cite{Nak201001}. En el \'ultimo rengl\'on de la tabla 4.2 se observa que el 
\'unico conjunto de intervalos de masa con los que se obtuvieron resultados que
pudieron ajustarse a los datos experimentales son los reportados por Koide
~\cite{Fus199701}, es decir, la funci\'on deseabilidad tom\'o su valor m\'aximo,
lo que significa que todos los observables ajustados est\'an dentro del error
experimental y a los cuales les corresponde una $\chi^2=0.35$ calculada de la
manera reportada en PDG ~\cite{Nak201001}. Los par\'ametros que se utilizaron
para hacer el ajuste son las seis masas de los quarks y cinco par\'ametros
libres, de los cuales tres son fases. El ajuste con las masas reportadas por
Emmanuel-Costa tiene un valor de la funci\'on deseablidad mayor al obtenido con
las masas reportadas por Xing y un valor de $\chi^2$ menor. Sin embargo, ambos
ajustes son malos, ya que la funci\'on deseabilidad es menor a 12, y por lo
tanto, no todos los observables pudieron ser ajustados simult\'aneamente para un
mismo conjunto de par\'ametros.

{\tiny
\begin{table}[h!]\label{tra}
\caption{Valores te\'oricos obtenidos de $|V_{ij}|$ utilizando la texturas con 
cuatro ceros de Fritzsch (F4) para las matrices de masa presentadas en la tabla 
~\ref{t3masas} y los valores experimentales de $|V_{ij}|$.}
$$\ba{l|ccc|r}
|V_{ckm}|^{teo} & \mbox{Koide} & \mbox{Emmanuel-Costa} & \mbox{Xing} & 
|V_{ckm}|^{exp}\\ \hline
|V_{ud}| & 0.97426 & 0.97412 & 0.97414 & 
0.97428\pm{0.00015} \\
|V_{us}| & 0.22537 & 0.22599 & 0.22589 & 
0.2253\pm{0.0007} \\
|V_{ub}| & 0.00343 & 0.00335 & 0.00187 & 
0.00347^{+0.00016}_{-0.00012} \\
|V_{cd}| & 0.22523 & 0.22588 & 0.22589 & 
0.2252\pm{0.0007} \\
|V_{cs}| & 0.97345 & 0.97343 & 0.97409 & 
0.97345^{-0.00015}_{-0.00016} \\
|V_{cb}| & 0.04082 & 0.03748 & 0.01090 & 
0.0410^{+0.0011}_{-0.0007} \\
|V_{td}| & 0.00869 & 0.00813 & 0.00184 & 
0.00862^{+0.00026}_{-0.00020} \\
|V_{ts}| & 0.04003 & 0.03674 & 0.01091 & 
0.0403^{+0.0011}_{-0.0007} \\
|V_{tb}| & 0.999160 & 0.999291 & 0.999939 & 
0.999152^{+0.000030}_{-0.000045} \\
|V_{td}|/|V_{ts}| & 0.2171 & 0.2213 & 0.1685 & 0.21\pm{0.04} \\
\chi^2& 0.350 & 40.456 &2752 \\
F_d & 12 & 11.938 & 11.381 & 12 \\
\ea
$$\end{table} }
  
En la tabla 4.3 se muestran los valores ajustados para los observables de
la violaci\'on de $CP$, solo los datos de la primera columna est\'an en el
intervalo de error reportado experimentalmente. En la ecuaci\'on ~(\ref{supuni})
del cap\'itulo anterior se presenta la forma de la matriz $V_{ckm}$ te\'orica
la cual es unitaria. La suma de dos \'angulos internos de los tri\'angulo
unitario se determina el valor del tercero. Los valores presentados
en la tabla 4.3 de los \'angulos $\alpha$ y $\gamma$ se obtuvieron a 
partir de las ecuaciones ~(\ref{alpha}) y ~(\ref{gamma}), el valor del \'angulo
$\beta$ de la condici\'on de unitariedad, con lo que se asegura la existencia
del tri\'angulo unitario y que la suma de los tres \'angulos deber ser 180. Por
\'ultimon el valor te\'orico reportado para el invariante de Jarlskog se
calcul\'o de la ecuaci\'on ~(\ref{idj}).
 
{\tiny
\begin{table}[h!]\label{tra2}
\caption{Valores te\'oricos obtenidos de $\alpha$, $\beta$, $\gamma$ y $J$ para 
las texturas con cuatro ceros de Fritzsch (F4) para las matrices de masa presentadas en la tabla ~\ref{t3masas} y los valores experimentales.}
$$\ba{l|ccc|r}
 & \mbox{Koide} & \mbox{Emmanuel-Costa} & \mbox{Xing} &\mbox{exp}\\ \hline
\alpha & 87.9 & 84.6 & 84.6 & 89.0^{+4.4}_{-4.2}  \\
\gamma & 70.7 & 72.8 & 48.0 & 73^{+22}_{-25}      \\
\beta  & 21.3 & 22.6 & 47.4 & 21.1495^{+0.905}_{-0.8787} \\
J (10^{-5})& 2.901 & 2.64 & 0.333 & 
2.91^{+0.19}_{-0.11} \\
\ea
$$\end{table} }

Los par\'ametros de entrada con los que se obtuvieron los valores reportados
en las tablas 4.2 y 4.3 se presentan en la tabla 4.4. Los intervalos de las
masas son los presentados en la tabla ~\ref{t3masas} y los intervalos para los
par\'ametros $D_q$ son escogidos de tal manera que los elementos de la matriz
sim\'etrica $\tilde M_q$ de la ecuaci\'on ~(\ref{matsim}) sean reales. Los
valores de las fases en los \'ultimos tres renglones son libres y sin relaci\'on
alguna, sus intervalos son de $0\leq f_i\leq 360$. En la tabla 4.4, enter todas
las soluciones en las que se obtuvo un valor de la funci\'on deseabilidad de
doce, utilizando las masas de Koide, se presenta el conjunto al que le
corresponde un valor menor de $\chi^2$. En las otras dos columnas se presentan
los conjuntos de valores para los par\'ametros de entrada con los que se obtuvo
el mayor valor de la funci\'on deseabilidad.

{\tiny
\begin{table}[h!]\label{trare}
\caption{Valores de los par\'ametros con los que se obtienen los resultados de
los ajustes n\'umericos presentados en las tablas 4.2 y 4.3.}
$$\ba{l|ccc}
 & \mbox{Koide} & \mbox{Emmanuel-Costa} & \mbox{Xing} \\ \hline
m_u & 0.00274 & 0.00199 & 0.00176 \\
m_c & 0.61606 & 0.55000 & 0.66717 \\
m_t & 169.567 & 170.661 & 169.954 \\
D_u & 159.783 & 164.387 & 106.365 \\
m_d & 0.00403 & 0.00210 & 0.00485 \\
m_s & 0.10409 & 0.07499 & 0.10495 \\
m_b & 2.97899 & 2.87811 & 2.80000 \\
D_d & 2.80023 & 2.76944 & 1.72287 \\
f_1 & 75.60   & 135.71  & 10.79   \\
f_2 & 74.97   & 135.74  & 104.53  \\
f_3 & 75.77   & 136.73  & 104.73  \\
\ea
$$\end{table} }

Dado que con las masas de Koide se pudieron ajustar simult\'aneamente los
m\'odulos de la matriz de mezcla y los par\'ametros de la violaci\'on de $CP$
te\'oricos a los datos experimentales cuando las matrices de masa presentan
texturas de Fritzsch con dos ceros en ambos sectores (F4), se usan estos
intervalos de masa para intentar obtener resultados que se ajusten a los datos
experimentales utilizando texturas diferentes. Los modelos que se usaron para
las matrices de masa fueron las texturas de Fritzsch con cinco ceros (F2F3 y
F3F2) y la textura de Fritzsch con cuatro ceros (F4), y las texturas I y IV de
Ramond, las cuales son presentadas en la tabla 4.1. Los cinco resultados se
obtuvieron utilizando algoritmos gen\'eticos durante 2160 generaciones. 

{\tiny
\begin{table}[h!]\label{trbb}
\caption{Valores te\'oricos obtenidos de $|V_{ij}|$ utilizando las texturas 
presentadas en la tabla 4.1  para las matrices de masa de Koide presentadas en 
la tabla ~\ref{t3masas}.}
$$\ba{l|ccccc|r}
|V_{ckm}|^{teo} & \mbox{F4-90} & \mbox{F2F3} & \mbox{F3F2 (II)} & 
\mbox{F2R1 (IV)} & \mbox{R1F2 (I)} & 
|V_{ckm}|^{exp}\\ \hline
|V_{ud}| & 0.97415 & 0.97415 & 0.97047 & 0.97263 & 0.96233 & 
0.97428\pm{0.00015} \\
|V_{us}| & 0.22591 & 0.22590 & 0.22460 & 0.22550 & 0.22460 & 
0.2253\pm{0.0007} \\
|V_{ub}| & 0.00055 & 0.00026 & 0.08803 & 0.05601 & 0.15320 & 
0.00347^{+0.00016}_{-0.00012} \\
|V_{cd}| & 0.22590 & 0.22590 & 0.22569 & 0.22590 & 0.22773 & 
0.2252\pm{0.0007} \\
|V_{cs}| & 0.97409 & 0.97415 & 0.97418 & 0.97414 & 0.97366 & 
0.97345^{-0.00015}_{-0.00016} \\
|V_{cb}| & 0.01083 & 0.00151 & 0.00594 & 0.00375 & 0.01072 & 
0.0410^{+0.0011}_{-0.0007} \\
|V_{td}| & 0.00244 & 0.00026 & 0.08521 & 0.05437 & 0.14849 & 
0.00862^{+0.00026}_{-0.00020} \\
|V_{ts}| & 0.01056 & 0.00152 & 0.02291 & 0.01395 & 0.03915 & 
0.0403^{+0.0011}_{-0.0007} \\
|V_{tb}| & 0.999941 & 0.999999 & 0.996100 & 0.998423 & 0.988138 & 
0.999152^{+0.000030}_{-0.000045} \\
|V_{td}|/|V_{ts}| & 0.2307 & 0.1685 & 3.7184 & 3.8969 & 3.7925 & 0.21\pm{0.04} \\
\chi^2& 2922.87 & 4579.7  & 374586  & 141387  & 1260082 \\
F_d   & 11.662  & 11.567  & 10.046  & 10.310  & 9.583   & 12 \\
\ea
$$\end{table} }
Se observa en el \'ultimo rengl\'on de la tabla 4.4 que con ninguna textura se
pudo obtener resultados que se ajusten satisfactoriamente a los datos
experimentales. En la primera columna se presentan los resulatados de imponer
que la fase $f_1$ sea de 90 grados y las otras dos fases tomen un valor de cero
usando el modelo de 4 ceros de Fritzsch la funci\'on deseabilidad no alcanz\'o
su valor m\'aximo, por esto el ajuste es malo. Al resto de las texturas se les
permiti\'o como par\'ametro libre la matriz de fases y no ajustaron los datos
experimentales, teniendo un mejor resultado aquellas donde la matriz de masa del
sector $u$ presenta la textura de dos ceros de Fritzsch, y de ellas la textura
de cinco ceros de Fritzsch (F2F3) ajusta mejor (la funci\'on deseabilidad es
mayor y el valor de $\chi^2$ menor) que la cuarta textura de Ramond (F2R1). 

En la tabla 4.6 se presentan los valores de los par\'ametros de la
violaci\'on de $CP$ correspondientes a las matrices de mezcla con las que se
obtuvo la tabla 4.5 y fueron calculados de la misma manera antes
explicada.

{\tiny
\begin{table}[h!]\label{trba}
\caption{Valores te\'oricos ajustados de los observables $\alpha$, $\beta$,  
$\gamma$ y $J$ para las cinco texturas presentadas en la tabla 4.1
utlilizando las masas de Koide.}
$$\ba{l|ccccc|r}
 &\mbox{F4-90}&\mbox{F2F3}&\mbox{F3F2 (II)}&\mbox{F2R1 (IV)}&\mbox{R1F2 (I)}&
\mbox{exp}\\ \hline

\alpha & 84.6 & 84.6 & 0.82 & 0.87 & 0.91 & 89.0^{+4.4}_{-4.2}  \\
\gamma & 82.7 & 48.0 & 64.9 & 76.7 & 73.1 & 73^{+22}_{-25}      \\
\beta  & 12.6 & 47.4 & 114.2 & 102.4 & 106.0 & 21.1495^{+0.905}_{-0.8787} \\
J (10^{-5})& 0.006   & 10.38 & 30.73 & 4.478 & 34.446 & 
2.91^{+0.19}_{-0.11} \\
\ea
$$\end{table} }


Los par\'ametros de entrada con los que se obtuvieron los valores reportados
en las tablas 4.5 y 4.6 se presentan en la tabla 4.7.
Los valores de las fases en los \'ultimos tres renglones  de la primera columna
(F4-90) corresponden a las restricciones impuestas al espacio de los
par\'ametros de entrada. El valor de cero para $D_d$ y $D_u$ en la segunda
(F2F3) y tercera (F3F2) columna respectivamente corresponden a las restricciones
propias de ambos modelos de texturas. El valor de cero para $D_d$ y $D_u$ en la
tercera (F3F2) y cuarta (F2R1) columna respectivamente corresponden a los
valores encontrados por el AG\footnote{Los par\'ametros en los que aparece * no
son libres, y est\'an determinados a partir de las masas de los quarks.}. En la
\'ultima columna se presentan los intervalos de masa reportados por Koide.

{\tiny
\begin{table}[h!]\label{trare2}
\caption{Valores de los par\'ametros con los que se obtienen los resultados de
los ajustes n\'umericos presentados en las tablas 4.5 y 4.6} 
%~\ref{trb2} y ~\ref{trba}}
$$\ba{l|ccccc|c}
 &\mbox{F4-90}&\mbox{F2F3}&\mbox{F3F2 (II)}&\mbox{F2R1 (IV)}&\mbox{R1F2 (I)}
&\mbox{masa de Koide} \\ \hline
m_u   & 0.00188 & 0.00193 & 0.00275 & 0.00275 & 0.00275 & 
0.00233^{+0.00045}_{-0.00042}\\
m_c   & 0.73296 & 0.61834 & 0.73299 & 0.61600 & 0.61600 & 
0.677^{+0.056}_{-0.61}\\
m_t   & 173.461 & 178.497 & 168.000 & 193.999 & 168.408 & 
181\pm13\\
D_u   & 109.954 &   5.315 & 0.0     & 0.00000 & *       & \\
m_d   & 0.00503 & 0.00528 & 0.00529 & 0.00529 & 0.00529 & 
0.00469^{+0.0060}_{-0.0066}\\
m_s   & 0.10519 & 0.10499 & 0.08040 & 0.08040 & 0.08040 & 
0.0934^{+0.0118}_{-0.0130}\\
m_b   & 2.89000 & 2.89000 & 3.10999 & 2.89000 & 3.11000 & 
3\pm0.11\\
D_d   & 1.79936 & 0.0     & 0.00000 & *       & 0.00000 & \\
f_1   & 90.0    & 269.94  & 1.47    & 205.87  & 98.2    & \\
f_2   &  0.0    & 0.001   & 62.69   & 268.73  & 162.5   & \\
f_3   &  0.0    & 0.076   & 62.93   & 0.00    & 0.00    & \\
\ea
$$\end{table} }


Como se sabe en la naturaleza solo hay una fase responsable de la violaci\'on de
$CP$, adem\'as los observables de la matriz de mezcla de los quarks son 
invariantes ante el refasamiento de los quarks. En el ajuste num\'erico 
presentado en la tabla 4.4 se ve claramente que en los tres casos dos 
fases son iguales entre s\'i, por lo que se puede absorber una fase en los 
campos de los quarks qued\'andose con una sola fase responsable de la 
violaci\'on de $CP$ en el modelo. Por esta raz\'on se hizo un ajuste num\'erico 
con las masas de Koide haciendo que una fase sea igual a cero, $f_2=0$, y las
otras dos fases sean iguales entre s\'i, $f_1=f_3$, pero variando libremente.
Para lo anterior se us\'o la textura con cuatro ceros de Fritzsch (F4) y
las matrices de masa que se obtienen de este ajuste son las siguientes, 
\be
M_u=\left(\ba{ccc} 0&0.16877&0\\
0.16877&171.67256&44.34728\\
0&44.34728&10.80211\ea\right),
\ee
donde la matriz de masa para los quarks $u$ es real y toda la violaci\'on de 
$CP$ viene de la matriz de masa compleja del sector $d$ 
\be
M_d=\left(\ba{ccc} 0&0.12462e^{-0.80017i}&0\\
0.12462e^{0.80017i}&2.75342&0.72220e^{0.80017i}\\
0&0.72220e^{-0.80017i}&0.08719\ea\right).
\ee
Con estas matrices de masa la func\'on deseabilidad toma su valor m\'aximo, 
$F_d=12$, por lo tanto todos los observable se ajustan simult\'aneamente a los
datos experimentales.

