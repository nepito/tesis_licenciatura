\chapter{Conclusiones}

Las texturas de Fritzsch con 2 ceros en ambos sectores permiten un excelente 
ajuste ($F_d=12$ y $\chi^2=0.350$) a los datos experimentales, utilizando los
 valores de los intervalos de masas para los quarks reportados por Koide. Los 
nueve m\'odulos de los elementos de la matriz de mezclas, los tres \'angulos del
tri\'angulo unitario y el invariante de Jarlskog obtenidos est\'an de acuerdo 
con los valores centrales reportados ~\cite{Nak201001}. Las texturas no permiten
hacer predicciones para la matriz de mezclas ya que se tienen m\'as de cuatro 
par\'ametros en la matriz de mezcla te\'orica.

El ajuste num\'erico sugiere que los tres pares de texturas de Robert 
~\cite{rrr} estudiadas en el presente trabajo (F3F2, F2R1 y R1F2) son reducidas 
de texturas con cinco ceros a texturas con seis ceros, en donde el par\'ametro 
libre $D_q$ es ajustado en la optimizaci\'on a cero. 

Con el conjunto de masas de los quarks reportado por Koide, la textura con
cuatro ceros (F4) en las matrices de masa y las restricciones en el espacio de
los par\'ametros de las fases no se consigui\'o un resultado \'optimo, ya que la
funci\'on deseabilidad $F_d=11.662$ no es 12 y la $\chi^2=2922.9$ es muy grande.
Para la textura F4 no se pudo encontrar un resultado compatible con el maximal
en el invariante de Jarlskog.

A partir de los datos presentados en la tabla 4.2 se concluye que no se 
pudo ajustar a los datos experimentales de la mezcla de los quarks y 
violaci\'on de $CP$ con las masas reportadas por Xing y Emmanuel-Costa 
utilizando texturas de Fritzsch con dos ceros en ambos sectores. 

A partir de las masas de Koide no se pudo obtener que la matriz de mezcla, los
\'angulos del tri\'angulo unitario y el invariante de Jarlskog ajustaran de
manera simult\'anea con los otros cuatro pares de texturas de cinco ceros
estudiados.

La optimizaci\'on con AG no es un m\'etodo exhaustivo por lo tanto no se puede
asegurar que los modelos en los que no se pudo obtener el valor m\'aximo de la
funci\'on deseabilidad no sean compatibles con los datos experimentales. 
Es posible que si las funciones
deseabilidad individuales sean iguales a la propuesta por Derringer, y en lugar
de que la funci\'on deseabilidad sea la suma de las doce funciones individuales
sea la media geom\'etrica, se obtengan mejores resultados. Otra posible mejora
es que la selecci\'on no sea elitista, ya que esta elecci\'on pudo crear 
c\'omulos grandes de soluciones optimizando un m\'inimo local.

Una propuesta para darle continuidad a este trabajo es agregar hip\'otesis de
nueva f\'isica, como la presntada en el cap\'itulo 3 \cite{Bra200601}, a los
modelos con los que no se puedan obtener resultados que
se ajusten a los datos experimentales, con la intenci\'on de investigar si con
ellas si es posible. Otra propuesta est\'a dirigida a la mejora del
m\'etodo. Por ejemplo, adem\'as de comparar la eficiencia del AG actual contra
aqu\'el donde se apliquen las propuestas del p\'arrafo anterior, se podr\'ian
obtener los intervalos en los par\'ametros de entrada para los cuales la
funci\'on deseabilidad sea 12.
