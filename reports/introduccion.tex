\chapter{Introducci\'on}

%%%%%%%%%%%%%%%%%%%%%%%%%%%%%%%%%%%%%%%%%%%%%%%%%%%%%%%%%%%%%%%%%%%%%%%%%%%
\section{Justificaci\'on}
A pesar del enorme progreso experimental, el origen de la masa de los
fermiones es a\'un una pregunta fundamental no resuelta en f\'isica de 
part\'iculas. Generalmente, en el contexto del modelo est\'andar electrod\'ebil
(ME) las matrices de las masas de los quarks $M_u$ y $M_d$ son complejas (36 
par\'ametros reales), como se explica en el cap\'itulo 2, y tienen que ajustar 
diez par\'ametros f\'isicos: las seis masas de los quarks y los cuatro 
par\'ametros de la matriz unitaria de Cabibbo-Kobayashi-Maskawa, $V_{ckm}$, los
cuales son presentados en el cap\'itulo 3. El problema  tradicional del modelo 
de las matrices de masa de los quarks va m\'as all\'a del modelo est\'andar y
supone formas espec\'ificas para las matrices $M_{u,d}$ las cuales, despu\'es de
la diagonalizaci\'on, proporcionan los eigenvalores y los par\'ametros de 
mezcla. En el intento de entender la estructura del sabor escondida en la matriz
de masa de los fermiones se proponen texturas, que son matrices de masa con
algunos ceros impuestos a los elementos de matriz. Suponer que hay simetr\'ia
del sabor oculta en las texturas de las matrices de masa de los quarks podr\'ia
proporcionar sugerencias de la din\'amica en la generaci\'on de masa de los
quarks y de la violaci\'on de $CP$ (tema presentado en el cap\'itulo en el
cap\'itulo 2) en un marco te\'orico m\'as fundamental ~\cite{Fri200001}.
 Entonces, dadas las matrices $M_u$ y $M_d$, es esencial
distinguir ceros que no tengan contenido f\'isico de aquellos otros que implican
restricciones f\'isicas en el espacio de los par\'ametros. El reto de un modelo
de texturas es ajustarse correctamente a los datos experimentales.

En este trabajo se estudiar\'an algunas texturas en las matrices de masa que 
pueden reproducir satisfactoriamente al fen\'omeno de la mezcla de los quarks. 
En la literatura se ha mostrado ~\cite{Mah200901} que las texturas de Fritzsch 
con 6 ceros en las matrices de masa son totalmente descartables ya que no se
pueden obtener resultados que concuerden con los datos experimentales, mientras
que las texturas de 5 ceros de Fritzsch en las matrices de masa no pueden ser 
descartadas completamente. Con la intenci\'on de obtener un entendimiento de las
mezclas de los quarks surge el problema de encontrar la estructua de la textura 
m\'as simple que sea compatible con el fen\'omeno de mezcla de los quarks. En 
vista de la ausencia de una justificaci\'on te\'orica para las matrices de masa 
como las de Fritzsch, se vuelve esencial desde el punto de vista 
fenomenol\'ogico considerar matrices de masa diferentes a las de Fritzsch, tanto
para los quarks como para los leptones.

Con la mejora en la precisi\'on de los datos experimentales los modelos de cinco
y seis ceros tienen gran dificultad en ajustarse a los datos ~\cite{Bra199902}. 
Usando criterios de naturalidad y simplicidad Fritzsch y Xing ~\cite{Fri200001} 
encontraron desfavorables a los modelos III y V de las texturas de
Robert, Ramond y Ross ~\cite{rrr} las cuales se muestran en la tabla ~\ref{t2}.
Con los datos experimentales disponibles en el a\~no 2008 se encontr\'o que 
ninguna textura de seis ceros puede reproducir los datos experimentales y que la
\'unica de cinco ceros que s\'i puede es la textura de Fritzsch con la matriz de
masa del sector $u$ con dos ceros y el sector $d$ con tres ceros 
~\cite{Mah200901}. Las texturas anteriores se proponen sin ninguna 
consideraci\'on f\'isica m\'as alla de la busqueda de modelos que reproduzcan 
los datos experimentales. Una suposici\'on f\'isica que puede ser impuesta a las
texturas, con la intenci\'on de obtener un modelo de ceros consistente con tal 
suposici\'on, es el maximal en el invariante de Jarlskog; dicha suposici\'on se 
explica en el cap\'itulo 3.

%%%%%%%%%%%%%%%%%%%%%%%%%%%%%%%%%%%%%%%%%%%%%%%%%%%%%%%%%%%%%%%%%%%%%%%%%%%%%
\section{Matrices de masa}
La densidad lagrangiana del modelo est\'andar electrod\'ebil (ME), que es el 
tema del siguiente cap\'itulo, puede ser expresada con cuatro t\'erminos  
\be\label{lme3f}
\mathcal{L}_{ME}=\mathcal{L}_{din}+\mathcal{L}_b+\mathcal{L}_{\Phi}
+\mathcal{L}_{Yuk},
\ee
los cuales representan las densidades lagrangianas de los fermiones, de los 
bosones, del campo de Higgs y la de Yukawa, respectivamente. Los dos primeros 
t\'erminos proporcionan las interacciones, existentes en el ME, de los bosones 
de norma con los fermiones, adem\'as de los t\'erminos cin\'eticos de ambos. 
Para prorcionar masa a los bosones y fermiones se agrega el tercer y cuarto
t\'ermino, la densidad lagrangiana del campo de Higgs, $\Phi$, y la de Yukawa.
En el modelo est\'andar el mecanismo de Higgs es el responsable de que los 
campos de norma y los fermiones adquieran masa ~\cite{Gri198701}. Despu\'es del 
rompimiento de simetr\'ia, la lagrangiana de Yukawa proporciona masa a los 
fermiones mediante el acoplo con el campo de Higgs, con lo que se obtienen las
masas de Dirac.

En la naturaleza existen tres generaciones de fermiones con id\'enticas 
caracter\'isticas, excepto la masa. En el ME no hay una explicaci\'on para 
esta propiedad. Los fermiones elementales conocidos est\'an divididos en 
dos categor\'ias, los quarks y los leptones. Una diferencia que los distingue 
es que los quarks participan en todas las interacciones (fuerte, 
electromagn\'etica, d\'ebil y gravitacional) mientras que los leptones no 
son afectados por la interacci\'on fuerte. En el ME de tres generaciones las 
masas de los quarks es una matriz compleja de 3 $\times$ 3, y cuya forma no es 
proporcionada por el ME, como se ver\'a en el pr\'oximo cap\'itulo.

En la base donde la teor\'ia es invariante de norma, los acoplos de Yukawa
no necesariamente son diagonales, por lo tanto los quarks no son estados 
propios de la masa ya que los t\'erminos de masa acoplan a una misma familia de
quarks de diferente generaci\'on. Cuando se pasa a una base donde la matriz 
de masa de los quarks es diagonal, la base de los quarks f\'isicos, las 
corrientes cargadas ya no son diagonales, los quarks $u$ y los quarks $d$  de 
diferentes generaciones se mezclan. La matriz no diagonal, responsable de la 
mezcla de los quarks, es la matriz de Cabibbo-Kobayashi-Maskawa, denotada como 
$V_{ckm}$, y es una manera de tener la violaci\'on de $CP$ en el modelo 
est\'andar.
 
La forma funcional de la matriz de mezcla se define como ~\cite{Rod200101} y 
~\cite{Koi200601}
\be
V_{ckm}=U^{u\dag}_LU^{d}_L=\left(\ba{ccc} V_{ud}&V_{us}&V_{ub}\\
V_{cd}&V_{cs}&V_{cb}\\ V_{td}&V_{ts}&V_{tb}\ea\right),
\ee
donde las matrices $U^q_L$ son las matrices que diagonalizan a las matrices de 
masa de los quarks $u$ y $d$, como se explica en el siguiente cap\'itulo.
%%%%%%%%%%%%%%%%%%%%%%%%%%%%%%%%%%%%%%%%%%%%%%%%%%%%%%%%%%%%%%%%%%%%%%%%%%%%
\section{Objetivo}
El objetivo del presente trabajo es poner a prueba la capacidad predictiva de
texturas en las matrices de masa para el sector de los quarks utilizando
algoritmos gen\'eticos. Para lograrlo primero se obtendr\'an los valores de los
m\'odulos de los elementos de la matriz de mezcla de los quarks en los
intervalos experimentales utilizando diferentes texturas. A las texturas con las
que se obtengan resusltados que se ajusten a los datos experimentales se les
probar\'a su compatibilidad con la hip\'otesis de maximal del invariante de
Jarlskog y se presentar\'an las matrices de masa con las que se obtengan
resultados de acuerdo con los datos experimentales. 

