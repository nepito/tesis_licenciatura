En este trabajo se aplican algor\'itmos gen\'eticos para dar soluci\'on a 
problemas con respuesta m\'ultiple con el objetivo de determinar el \'optimo
global dentro de una regi\'on experimental. 

Los resultados permiten inferir que algoritmos gen\'eticos tiene una alta 
efectividad para encontrar la regi\'on \'optima global, se obtiene muy buena 
soluci\'on aunque ne se haya explorado todo el espacio de b\'usqueda.

Por lo tanto se concluye que en situaciones similares a la resuelta en este 
trabajo es superior algoritmos gen\'eticos.

La optimizaci\'on de un proceso de respuesta m\'ultiple se refiere a determinar
los valores de las variables controlables en el que todas las caracter\'isticas
de calidad cumplan con los requerimientos de la mejor manera posible.

Los m\'etodos de optimizaci\'on se eval\'uan tomando como medidas de 
desempe\~no, su efectividad para encontrar un \'optimo global y las limitaciones
que tiene cada m\'etodo para poder aplicarse, y as\'i establecer ventajas y 
desventajas de cada uno de ellos, para obtener que m\'etodo de \'optimizaci\'on
es m\'as eficiente para dar una soluci\'on a problemas con respuestas 
m\'ultiples.

\subsection{Definici\'on del problema}
Sea $\vec Y=(y_1,y_2,...,y_n)$ un vector en $R^n$ que representa las $n$ 
caracter\'isticas de calidad a optimizar de un proceso, en el que los valores de
sus componentes dependen del vector $\vec X=(x_1,x_2,...,x_k)$, donde $k$ 
representa el n\'umero de factores controlables en el proceso.

El problema consiste en determinar el vector $XR^k$, tal que optimice la 
funci\'on vectorial
$$Y_i=f_1(x_1,x_2,...,x_k).$$

Diversos dise\~nos de experimentos involucran m\'a de una respuesta. Lo t\'ipico
seg\'un De la Vara(2002), es que la calidad del producto dependa del valor que
asume m\'as de una de sus propiedades. Muchas veces solo una de las 
caracter\'isticas del proceso hace que las otras propiedades se vean afectadas
y el resultado es un proceso con pero calidad global que antes. De aqu\'i la 
importancia de contar con t\'ecnicas que sirvan para optimizar de manera
simult\'aneaa todas las respuestas de interes. (Guti\'errez {\it et al, 2004}.

\subsection{Problema de respuesta m\'ultiple}
Cuando se tienen varias respuestas a considerar, este problema es llamado como
problema de respuesta m\'ultiple o problemas de optimizaci\'on simult\'anea. 

La consideraci\'on de respuestas m\'ultiples implica primera una constricci\'on 
apropiada de un modelo de superficie de respuestapara cada respuesta y despu\'es
tratar de encontrar un conjunto de condiciones de operaci\'on que en alg\'un 
sentido optimiza todas las respuestas o al menos las mantiene en intervalos 
deseados.

Existen m\'etodos de optimizaci\'on para resolver problemas de m\'ultiples 
respuestas como el de b\'usqueda exhaustiva, m\'etodo gr\'afico, simplex de 
Nelder y Mead, gradiente reducido generalizado y algoritmos gen\'eticos. Escoger
el m\'etodo de optimizaci\'on podr\'ia determinar si el problema se resuelve 
r\'apido o lento, e incluso, si el problema se resuelve o no. (Evaristo, 2009)

\subsection{Funci\'on deseabilidad}
Si por alg\'un motivo las propiedades pueden ser medidas en unidades 
consistentes, o incluso mejor, pueden ser expresadas como n\'umeros en una 
escala dimensional, entonces intentar combinar las escalas de medici\'on con
operadores aritm\'eticos es factible. Para desarrollar una transformaci\'on de
escalas, es necesario describir una escala com\'un para todas y para las cuales
algunas significancia f\'isica puede ser irrelevante, interpretada en t\'erminos
de deseabilidad para cualquier aplicaci\'on espec\'ifica (Harrington,1965).

La clase m\'as simple  de transformaciones se basa en que existe l\'imites de
especificaci\'on, los cuales vienen a ser la base y criterio inalterable de 
calidad. Afuera de estos l\'imites el valor de $d$ es cero, adentro de estos 
l\'imites es $O<d<1$ (Harrington, 1965).

La funci\'on deseabilidad primero convierte cada respuesta $y_i$ en una 
funci\'on de deseabilidad individual $d_i$ que var\'ia en un intervalo de 
$$0<d_i<1$$
donde si $d_i$ es igual a 1 significa que la correspondiente respuesta predicha
$y_i$ toma su valor m\'aximo deseable y si $d=0$ la respuesta $d_i$ predice un 
valor inaceptable  y en este caso significa que la deseabilidad global es cero.

\subsection{Funci\'on deseabilidad Derringer}
Suponiendo que las variables $Y_i$ por especializaciones inferiores y superios
$EI_i$ y $ES_i$ y su valor objetivo o nominal es $T_i$ se define la 
transformaci\'on $d_i$, suponiendo que exista un l\'imite de especificaci\'on, 
lo que se hace es tomar el valor objetivo $T_i$ igual al valor a partir del cual
se considera que no hay ganancia adicional en la calidad de la respuesta. Las
transformaciones est\'an dadas por

La funci\'on aptitud se define como la deseabilidad global, que est\'a dada por
la media geom\'etrica de las deseabilidades individuales, las cuales se obtienen
de la siguiente  manera
\begin{itemize}
\item<1-> Se obtienen los valores objetivos de cada respuesta, valores m\'inimos
de cada respuesta y los modelos de predicci\'on.
\item<2-> Se substituye los valores de las variables independientes en los 
modelos de predicci\'on de cada respuesta.
\item<3-> Se utilizan los valores que arrojan los modelos de predicci\'on para
especificar la funci\'on de deseabilidad individual para cada respuesta mediante
la funci\'on deseabilidad de Derringer, la cual se obtiene de la siguiente 
manera:
\end{itemize}

El problema puede requerir demasiado tiempo para converger debido a que el 
n\'umero de soluciones es exponencial al n\'umero de factores involucrados en el
proceso, su tiempo de soluci\'on o costo computacional tambi\'en se convierte en
exponencial.

\section{Tesis Evaro}
Optimizaci\'on es una importante herramienta en la toma de decisiones y en el 
an\'alisis se sistemas f\'isicos. Para optimizar se requiere definir una medida
cuantitativa a maximizar (o minimizar). Esta medida cuantitativa se le llama 
funci\'on costo. 

La funci\'on aptitud se define sobre la representaci\'on gen\'etica y mide la
calidad de la soluci\'on representada. En el presente problema, se determian el 
error en el ajuste de la funci\'on direcciional a los datos de los radares HF.

\section{Texturas}
Despu\'es de tener cuatro grupos de texturas con cinco y seis ceros en las 
matrices de masa, se quiere ver la compatibilidad con los datos de mezcla de
los quarks.
