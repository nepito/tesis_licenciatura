\chapter{C\'odigo}
C\'odigo en Matlab 8 ~\cite{matlab} de la funci\'on que se usa para poner a
prueba los modelos de texturas utilizando algoritmos gen\'eticos y las funciones
que se usan en ella.
\section{Algoritmo Gen\'etico}
\begin{verbatim}
%% Algoritmo Genetico
% Proporciona los valores de las masas, fases y parametros libres que mejor
% ajusten a los valores experimentales a los modelos de las texturas en las
% matrices de masa.
%Los parametros de entrada son n= el tamano de la poblacion.
%bmasas=intervalos de masas que se usan 1=Xing, 2=Koide y 3=Emmanuel-Costa
%bpdg  =datos experimenteles con los que se compararan los modelos,
%dependiendo del ano 
%bt = al modelos de la textura puesto a prueba 1=F4, 2=F2F3, 3=F3F2, 4=F2R1 
% y 5=R1F2.

function [poblacion,hist] = funalg_texturas_ultimate3(n,bmasa,bpdg,bt)

%% Parametros iniciales
% Se configua el Algoritmo definiendo parametros analogos en la Genatica
format long
 %Se crean la poblacion inicial y se cargan los datos experimentales.
%[poblacion,imu,imd,V,Vint,Ver,angu]=fundatexp(n,bmasa,bpdg);     
                             
generaciones = 15;            %Numero de generaciones que se dejara evolucionar
                              %al sistema
guardar      = 3;             %Numero de datos que se guardan
nC           = 20;            %Numero de cuates por parejas
migra        = 0;             %Generacion espontanea  
probabilidad = 1.0;           % probabilidad de que cada gen mute
hist=ones(guardar,12);        %Se crea la matriz hist. Donde se guada la 
                              %evolucion de la poblacion
par=11;                       %Parametros
obs=1;                        %Observables
                              %Maximo valor de la mutacion
mutacion = [0.00000250 0.0003 0.0115 0.00250 0.00000350 0.00075 0.00035 0.00250
            0.01 0.01 0.01]/5;
%% Poblaci�n inicial
% Se genera aleatoriamente una poblacion inicial de posibles soluciones.
% Las primeras 4 columnas representan los genes del sector u. Las
% siguientes 4 columnas representan los genes del sector d. Las ultimas 4
% columnas son las 3 fases posibles y la funcion desebilidad.
% Cada uno de los n individuos (renglones) representa una posible soluci�n.
%poblacion = rand(n,2);
imu=zeros(2,4);
imd=zeros(2,7);
[poblacion,imu,imd,V,Vint,Ver,ang]=fundatexp(n,bmasa,bpdg);

%% Iteraciones
% La poblacion de soluciones evoluciona mediante la cruza y mutacion
for ci= 1:guardar
    hw = waitbar(0,'Evolucionando...');
    for i = 1:generaciones
        temporal=zeros(2*(n*(nC+1)+migra),par+obs);
        %% Orden aleatorio de parejas
        % La formacion de parejas para la reproduccion es aletoria
        poblacion(:,par+1:obs+par) = rand(n,obs);
        poblacion = sortrows(poblacion,+(par+1));
        temporal(1:n,:) = poblacion;
%% Cruza
        for j=1:n/2
            mama=rand(nC,par+obs)>0.5;
            papa=~mama;
            temporal(n+1+   (2*nC*(j-1)):n+   (2*nC*(j-1))+nC,:) =
repmat(poblacion(2*j,:),nC,1).*mama + repmat(poblacion(2*j-1,:),nC,1).*papa;
            temporal(n+1+nC+(2*nC*(j-1)):n+nC+(2*nC*(j-1))+nC,:) =
repmat(poblacion(2*j,:),nC,1).*papa + repmat(poblacion(2*j-1,:),nC,1).*mama;
        end
 %Generacion Espontanea
        temporal(n*(nC+1)+1:n*(nC+1)+migra,:) = [repmat(imu(1,:),migra,1)+
repmat((imu(2,:)-imu(1,:)),migra,1).*rand(migra,4) repmat(imd(1,:),migra,1)+
repmat((imd(2,:)-imd(1,:)),migra,1).*rand(migra,7) rand(migra,obs)];
        s=n*(nC+1)+migra;
%% Mutacion
        %Se calcula la mutacion, los limites permitidos y se muta. 
        muta=(rand(s,par+obs)*2-1).*repmat([mutacion zeros(1,obs)],s,1);
        mayor=[repmat(imu(2,:),s,1)-temporal(1:s,1:4) repmat(imd(2,:),s,1)-
temporal(1:s,5:11) zeros(s,obs)];
        menor=[repmat(imu(1,:),s,1)-temporal(1:s,1:4) repmat(imd(1,:),s,1)-
temporal(1:s,5:11) ones(s,obs)];
        b=(rand(s,par+obs)<probabilidad).*(mayor>muta).*(menor<muta);
        temporal(s+1:2*s,:)=temporal(1:s,:)+b.*muta;
%% Calculo de la aptitud (o viabilidad)
        % La columna numero 12 de poblacion representa la aptitud (viabilidad) 
        % de la solucion, mayor es mejor.
        temporal=funaptult(temporal,V,Vint,ang,bt);
        temporald = unique(temporal,'rows');
        temporalb = sortrows(temporald,-12);
        s = min([size(temporalb,1),n]);
 %Seleccion
        poblacion = temporalb(1:s,:);
        waitbar(i/generaciones,hw)
    end
    close(hw)
    hist(ci,:)=poblacion(1,:);
end

\end{verbatim}


\section{Datos experimentales}
\begin{verbatim}
%% Datos experimentales
%Proporciona los valores experimentales de los observables (los modulos de 
%la matriz de mezcla y la fenomenologia de la violacion de CP) ademas de
%losintervalos permitidos para las masas, reportados por Xing, Koide y
%Emmanule-Costa. 
%Los parametros de entrada son
%n= El tamano de la poblacion.
%bmasa= Los intervalos de masa desados 1 Xing, 2 Koide y 3. Emmanuel-Costa
%bpdg = Los valores reportados en PDG dependiendo del a\~no 1=2008 y
%2=2010.
%Los valores de salida son poblacion=poblacion inicial.
%imu=Intervalos permitidos para los qurks del sector u.
%imd=Intervalos permitidos para los qurks del sector d.
%Los valores de experimentales centrales de los modulos de la matriz Vckm.
%Los intervalos experimentales de los modulos de la matriz Vckm.
%Los valores experimentales de los observables de la violacion de CP.

function [poblacion,imu,imd,V,Vint,Ver,ang]=fundatexp(n,bmasa,bpdg)

imu=zeros(2,4);
imd=zeros(2,7);
if bmasa == 1
%%%%%%%%%%%%%%%%%%%%%%%%%%%%%%%%%%%%%%%%%%%%%%%%%%%%%%%%%%%%%%%%%%%%%%%%%%%
%%%%%%%%%%%%%%%%%%%%%%%%%%%%% Xing 2008 %%%%%%%%%%%%%%%%%%%%%%%%%%%%%%%%%%%
  imu(:,1)=0.00127+[-0.00042;0.00050];       %masa del quark u 1.4+0.6-0.5
  imd(:,1)=0.00290+[-.00119;0.001240];       %masa del quark d 2.8+-0.7
  imd(:,2)=0.055+[-0.0150;0.016];         %masa del quark s 60+15-19
  imu(:,2)=0.619+[-0.084;0.084];          %masa del quark c 0.64+0.07-0.09
  imd(:,3)=2.89+[-0.09;0.09];          %masa del quark b 2.84+0.17-0.08
  imu(:,3)=171.7+[-3;3];        %masa del quark t 170.1+-2.3
end
if bmasa == 2
%%%%%%%%%%%%%%%%%%%%%%%%%%%%%%%%%%%%%%%%%%%%%%%%%%%%%%%%%%%%%%%%%%%%%%%%%%%
%%%%%%%%%%%%%%%%%%%%%%%%%% Koide 2010 %%%%%%%%%%%%%%%%%%%%%%%%%%%%%%%%%%%%% 
 imu(:,1)=0.00233 +[-0.00045;0.00042];       %masa del quark u 1.4+0.6-0.5
 imd(:,1)=0.00469 +[-0.00066;0.00060];       %masa del quark d 2.8+-0.7
 imd(:,2)=0.0934  +[-0.0130;0.0118];         %masa del quark s 60+15-19
 imu(:,2)=0.677   +[-0.061;0.056];          %masa del quark c 0.64+0.07-0.09
 imd(:,3)=3       +[-0.11;0.11];          %masa del quark b 2.84+0.17-0.08
 imu(:,3)=181     +[-13;13];        %masa del quark t 170.1+-2.3
end
if bmasa == 3
%%%%%%%%%%%%%%%%%%%%%%%%%%%%%%%%%%%%%%%%%%%%%%%%%%%%%%%%%%%%%%%%%%%%%%%%%%%
%%%%%%%%%%%%%%%%%%%%%%%%% Emmanuel-Costa 2009 %%%%%%%%%%%%%%%%%%%%%%%%%%%%%
 imu(:,1)=0.0014  +[-0.0005;0.00060];       %masa del quark u 1.4+0.6-0.5
 imd(:,1)=0.00280 +[-.0007;0.0007];       %masa del quark d 2.8+-0.7
 imd(:,2)=0.060   +[-0.0190;0.015];         %masa del quark s 60+15-19
 imu(:,2)=0.64    +[-0.09;0.07];          %masa del quark c 0.64+0.07-0.09
 imd(:,3)=2.89    +[-0.08;0.17];          %masa del quark b 2.84+0.17-0.08
 imu(:,3)=170.1   +[-2.3;2.3];        %masa del quark t 170.1+-2.3
end

imu(:,4)=[0;sum(imu(:,3),1)/2];                %intervalo del Parametro libre
imd(:,4)=[0;sum(imd(:,3),1)/2];                %intervalo del Parametro Libre
imd(:,5)=2*pi*[0;1];        %Intervalo de los angulos
imd(:,6)=2*pi*[0;1];        %Intervalo de los angulos
imd(:,7)=2*pi*[0;1];        %Intervalo de los angulos 

%Modulos de la matriz de mezcla y sus errores
if bpdg == 1 
%%%%%%%%%%%%%%%%%%%%%%%%%%%%%%%%%%%%%%%%%%%%%%%%%%%%%%%%%%%%%%%%%%%%%%%%%%%
%%%%%%%%%%%%%%%%%%%%%%%%%%%%%% PDG 2008 %%%%%%%%%%%%%%%%%%%%%%%%%%%%%%%%%%%
 V(1,1) = 0.97419;
 V(1,2) = 0.2257;
 V(2,1) = 0.2256;
 V(2,2) = 0.97334;
 V(2,3) = 0.0415;   
 V(1,3) = 3.59/10^3;
 V(3,1) = 8.74/10^3;
 V(3,2) = 40.7/10^3;
 V(3,3) = 0.999133;

 Vint=zeros(3,3,2);
 Vint(1,1,1)=-0.00022;
 Vint(1,2,1)=-0.0010;
 Vint(1,3,1)=-0.00016;
 Vint(2,1,1)=-0.0010;
 Vint(2,2,1)=-0.00023;
 Vint(2,3,1)=-0.0011;
 Vint(3,1,1)=-0.00037;
 Vint(3,2,1)=-0.0010;
 Vint(3,3,1)=-0.00043;
 Vint(1,1,2)=0.00022;
 Vint(1,2,2)=0.0010;
 Vint(1,3,2)=0.00016;
 Vint(2,1,2)=0.0010;
 Vint(2,2,2)=0.00023;
 Vint(2,3,2)=0.0010;
 Vint(3,1,2)=0.00026;
 Vint(3,2,2)=0.0010;
 Vint(3,3,2)=0.00044;
 
 ang(1,1) = 3.05/10^5;         %PDG 2008
 ang(2,1) = 0.20/10^5;         %PDG 2008
 ang(1,2) = 88;                %PDG 2008
 ang(2,2) = 6.0;               %PDG 2008
 ang(1,3) = 77;                %PDG 2008
 ang(2,3) = 32;                %PDG 2008
 
end
if bpdg == 2
%%%%%%%%%%%%%%%%%%%%%%%%%%%%%%%%%%%%%%%%%%%%%%%%%%%%%%%%%%%%%%%%%%%%%%%%%%%
%%%%%%%%%%%%%%%%%%%%%%%%%%%%% PDG 2010 %%%%%%%%%%%%%%%%%%%%%%%%%%%%%%%%%%%%
    V(1,1) = 0.97428;
    V(1,2) = 0.2253;
    V(2,1) = 0.2252;
    V(2,2) = 0.97345;
    V(2,3) = 0.0410;
    V(1,3) = 3.47/10^3;
    V(3,1) = 8.62/10^3;
    V(3,2) = 40.3/10^3;
    V(3,3) = 0.999152;

    Vint=zeros(3,3,2);
    Vint(1,1,1)=-0.00015;
    Vint(1,2,1)=-0.0007;
    Vint(1,3,1)=-0.00012;
    Vint(2,1,1)=-0.0007;
    Vint(2,2,1)=-0.00016;
    Vint(2,3,1)=-0.0007;
    Vint(3,1,1)=-0.00020;
    Vint(3,2,1)=-0.0007;
    Vint(3,3,1)=-0.00045;
    Vint(1,1,2)=0.00015;
    Vint(1,2,2)=0.0007;
    Vint(1,3,2)=0.00016;
    Vint(2,1,2)=0.0007;
    Vint(2,2,2)=0.00015;
    Vint(2,3,2)=0.0011;
    Vint(3,1,2)=0.00026;
    Vint(3,2,2)=0.0011;
    Vint(3,3,2)=0.00030;

    ang(1,1) = 2.91/10^5;       %PDG 2010
    ang(2,1) = 0.19/10^5;       %PDG 2010
    ang(1,3) = 89;              %PDG 2010
    ang(2,3) = 4.4;             %PDG 2010
    ang(1,2) = 73;              %PDG 2010
    ang(2,2) = 25;              %PDG 2010 
end
 
Ver = max(abs(Vint(:,:,1)),Vint(:,:,2));

%Poblacion inicial
poblacion=[repmat(imu(1,:),n,1)+repmat(imu(2,:)-imu(1,:),n,1).*rand(n,4) 
repmat(imd(1,:),n,1)+repmat(imd(2,:)-imd(1,:),n,1).*rand(n,7)];  
\end{verbatim}

\section{Matriz de mezcla}
\begin{verbatim}
%% Matriz Vckm
%Se calcula la matriz de mezcla Vckm, y se unico valor de salida es el
%valor de la funcion deseabilidad, e.
%Los parametros de entrada son las matrices que diagonalizan a las matrices
%de masa del sector u y el sector d (Ou y Od)
%La matriz de fase, f, de la matriz de mezcla 
%El valor central de los modulos de la matriz de mezcla
%Los intervalos permitidos para los modulos de la matriz de mezcla
%Los valores centrales de los observables de la violacion de CP

function [e]=funmatvu(Ou,Od,f,V,Vint,ang)

%%%%%%%%%%%%%%%%%%%%%%%%%%%%%%%%%%%%%%%%%%%%%%%%%%%%%%%%%%%%%%%%%%%%%%%%%%%
%%%%%%%%%%%%%% Parametros de las rectas de funcion deseabilidad %%%%%%%%%%%
penPos=1./(1-V-Vint(:,:,2));
penPos(1,1)=1./(V(1,1)+Vint(1,1,1));
penPos(2,2)=1./(V(2,2)+Vint(2,2,1));
penPos(3,3)=1./(V(3,3)+Vint(3,3,1));
b1=(1-2*V-Vint(:,:,1)-Vint(:,:,2))./(1-V-Vint(:,:,2));
b1(1,1)=0;
b1(2,2)=0;
b1(3,3)=0;
b2=1./(1-V-Vint(:,:,2));
b2(1,1)=(2*V(1,1)+Vint(1,1,2)+Vint(1,1,1))/(V(1,1)+Vint(1,1,1));
b2(2,2)=(2*V(2,2)+Vint(2,2,2)+Vint(2,2,1))/(V(2,2)+Vint(2,2,1));
b2(3,3)=(2*V(3,3)+Vint(3,3,2)+Vint(3,3,1))/(V(3,3)+Vint(3,3,1));
%%%%%%%%%%%%%%%%%%%%%%%%%%%%%%%%%%%%%%%%%%%%%%%%%%%%%%%%%%%%%%%%%%%%%%%%%%%
%%%%%%%%%%%%%%%%%%%%%%%%%%%%%% Calculos %%%%%%%%%%%%%%%%%%%%%%%%%%%%%%%%%%%
VckmC = Ou*diag(exp(1i*f))*Od;
%Modulos de los elementos de la matriz de mezcla
Vckm  = abs(abs(VckmC));
%Invariante de Jarlskog
fas(1) = imag(VckmC(1,2)*conj(VckmC(1,3))*conj(VckmC(2,2))*VckmC(2,3));
%Angulo gamma del triangulo unitario
fas(2) = angle(-VckmC(3,1)*conj(VckmC(3,3))/(VckmC(1,1)*conj(VckmC(1,3))))*
180/pi;
%Angulo alpha del triangulo unitario
fas(3) = angle(-VckmC(1,1)*conj(VckmC(1,3))/(VckmC(2,1)*conj(VckmC(2,3))))*
180/pi;
%%%%%%%%%%%%%%%%%%%%%%%%%%%%%%%%%%%%%%%%%%%%%%%%%%%%%%%%%%%%%%%%%%%%%%%%%%%
%%%%%%%%%%%%%%%%%%%%%%%%%% Funcion de deseabilidad %%%%%%%%%%%%%%%%%%%%%%%%
err=ones(3,3);
for i=1:3
    for j=1:3
        if (Vckm(i,j)<V(i,j)+Vint(i,j,1))
            err(i,j)=Vckm(i,j)*penPos(i,j)+b1(i,j);
        end
        if (Vckm(i,j)>V(i,j)+Vint(i,j,2))
            err(i,j)=-Vckm(i,j)*penPos(i,j)+b2(i,j);
        end
    end    
end
%%%%%%%%%%%%%%%%%%%%%%%%%%%%%%%%%%%%%%%%%%%%%%%%%%%%%
%%%%%%%%%%%%%%%%%%%%%%%%%%%%%%%%%%%%%%%%%%%%%%%%%%%%%
f1=1;
if (fas(3)<ang(1,3)-ang(2,3))
    f1=1/(360-ang(1,3)-ang(2,3))*fas(3)+1-(ang(1,3)-ang(2,3))/(360-ang(1,3)-
ang(2,3));
end
if (fas(3)>ang(1,3)+ang(2,3))
    f1=-1/(360-ang(1,3)-ang(2,3))*fas(3)+360/(360-ang(1,3)-ang(2,3));
end
%%%%%%%%%%%%%%%%%%%%%%%%%%%%%%%%%%%%%%%%%%%%%%%%%%%%%
f2=1;
if (fas(2)<ang(1,2)-ang(2,2))
    f2=1/(360-ang(1,2)-ang(2,2))*fas(2)+1-(ang(1,2)-ang(2,2))/(360-ang(1,2)-
ang(2,2));
end
if (fas(2)>ang(1,2)+ang(2,2))
    f2=-1/(360-ang(1,2)-ang(2,2))*fas(2)+360/(360-ang(1,2)-ang(2,2));
end
%%%%%%%%%%%%%%%%%%%%%%%%%%%%%%%%%%%%%%%%%%%%%%%%%%%%%
f3=1; %Invariante de Jarlskog
if (fas(1)<ang(1,1)-ang(2,1))
    f3=1/(1+ang(1,1)-ang(2,1))*(fas(1)+1);
end
if (fas(1)>ang(1,1)+ang(2,1))
    f3=1/(1+ang(1,1)-ang(2,1))*(2*ang(1,1)+1-fas(1));
end
%%%%%%%%%%%%%%%%%%%%%%%%%%%%%%%%%%%%%%%%%%%%%%%%%%%%%%%%%%%%%%%%%%%%%%%%%%%
%%%%%%%%%%%%%%%%%%%%%%%%%%%%%%%% Resultados %%%%%%%%%%%%%%%%%%%%%%%%%%%%%%%
d  =sum(sum(err,1),2);
e  = d+f1+f2+f3;
%%%%%%%%%%%%%%%%%%%%%%%%%%%%%%%%%%%%%%%%%%%%%%%%%%%%%%%%%%%%%%%%%%%%%%%%%%%
%%%%%%%%%%%%%%%%%%%%%%%%%%%%%%%%%%%%%%%%%%%%%%%%%%%%%%%%%%%%%%%%%%%%%%%%%%%
\end{verbatim}

\section{Aptitud}
\begin{verbatim}
%% Funcion deseabilidad
% Obtiene el valor de la funcion deseabilidad dependiendo de la textura
% presente en el modelo.
%Los parametros de entrada son masa=poblacion
%V   = el valor central de los modulos de la matriz de mezcla
%Vint= los intervalos de los modulos de la matriz de mezcla
%bt  = la textura puesta usada 

function [masa]=funaptult(masa,V,Vint,ang,bt)

s=size(masa,1);
%Textura de Fritzsch con 4 ceros
if bt == 1
    for p=1:s
        Ou=Fritzsch2(masa(p,1:3),masa(p,4));
        Od=Fritzsch2(masa(p,5:7),masa(p,8)); 
        f=masa(p,9:11);            
        [masa(p,12)]=funmatvu(Ou',Od,f,V,Vint,ang);
    end
end
%Textura de 5 ceros de Fritzsch
if bt == 2
    for p=1:s
        Ou=Fritzsch2(masa(p,1:3),masa(p,4));
        Od=Fritzsch3(masa(p,5:7),masa(p,8)); 
        f=masa(p,9:11);            
        [masa(p,12)]=funmatvu(Ou',Od,f,V,Vint,ang);
    end
end
%Textura de Ramond II
if bt == 3
    for p=1:s
        Ou=Fritzsch3(masa(p,1:3),masa(p,4));
        Od=Fritzsch2(masa(p,5:7),masa(p,8)); 
        f=masa(p,9:11);            
        [masa(p,12)]=funmatvu(Ou',Od,f,V,Vint,ang);
    end
end
%Textura de Ramond IV
if bt == 4
    for p=1:s
        Ou=Fritzsch2(masa(p,1:3),masa(p,4));
        Od=RRR1(masa(p,5:7),masa(p,8)); 
        f=masa(p,9:11);            
        [masa(p,12)]=funmatvu(Ou',Od,f,V,Vint,ang);
    end
end
%Textura de Ramond I
if bt == 5
    for p=1:s
        Ou=RRR1(masa(p,1:3),masa(p,4));
        Od=Fritzsch2(masa(p,5:7),masa(p,8)); 
        f=masa(p,9:11);            
        [masa(p,12)]=funmatvu(Ou',Od,f,V,Vint,ang);
    end
end
\end{verbatim}

\section{Texturas}
\subsection{Textura de Fritzsch}
\begin{verbatim}
%%Textura de Fritzsch con tres ceros
%Los valores de entrada de la funcion son las masas de los quarks del
%sector al que se le aplica dicho modelo. El resultado es la matriz que
%diagonaliza la la matriz de masas.
%Oq es la matriz diagonalizadora.
%mmqd es la matriz de masa diagonalizada.
%Elementos de la matriz de masas
function [Oq]=Fritzsch3(m)
Cq=m(1)-m(2)+m(3);
Aq=(m(1)*m(2)*m(3)/Cq)^(0.5);
Bq=(m(1)*m(2)+m(2)*m(3)-m(1)*m(3)-Aq^2)^(1/2);
%Matriz de masa
Mq=[0,Aq,0;Aq,0,Bq;0,Bq,Cq];
[Oq,mmud]=eig(Mq);
\end{verbatim}

\subsection{Textura de Fritzsch modificada}
\begin{verbatim}
%%Textura modificada de Fritzsch
%Los valores de entrada de la funcion son las masas de los quarks del
%sector al que se le aplica dicho modelo, y el valor del parametri libre 
%Dq. El resultado es la matriz que diagonaliza la la matriz de masas.
%Oq es la matriz diagonalizadora.
%mmqd es la matriz de masa diagonalizada.
%Elementos de la matriz de masas
function [Oq,Mq]=Fritzsch2(m,Dq)

Cq=(m(1)-m(2)+m(3))-Dq;
Aq=(m(1)*m(2)*m(3)/Cq)^(0.5);
Bq=(Cq*Dq+m(1)*m(2)+m(2)*m(3)-m(1)*m(3)-Aq^2)^(1/2);

%Matriz de masa
Mq=[0,Aq,0;Aq,Dq,Bq;0,Bq,Cq];
[Oq,mmud]=eig(Mq);
\end{verbatim}

\subsection{Textura de Ramond {\it et al.}}
\begin{verbatim}
%%Textura de Ramond, Robert y Ross
%Los valores de entrada de la funcion son las masas de los quarks del
%sector al que se le aplica dicho modelo. El resultado es la matriz que
%diagonaliza la la matriz de masas.
%Oq es la matriz diagonalizadora.
%mmqd es la matriz de masa diagonalizada.
%Calculo del elemento Cu
function [Oq]=RRR1(m)
    Cq=m(3);
    for i=1:100
        Cq=(Cq^2*(m(1)-m(2)+m(3))+Cq*(m(1)*m(2)+m(2)*m(3)-m(1)*m(3))-
m(1)*m(2)*m(3))^(1/3);
    end
    %Cq
    Dq=m(1)-m(2)+m(3)-Cq;
    Aq=(m(1)*m(2)*m(3)/Cq)^(0.5);
%Matriz de masa
    Mq=[0,Aq,0;Aq,Dq,0;0,0,Cq];
    [Oq,mmqd]=eig(Mq);
end
\end{verbatim}
